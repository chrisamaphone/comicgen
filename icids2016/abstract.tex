% Narrative generation is important
Narrative generation enables a range of opportunities for understanding 
the creative act of storytelling.
% but standard pipeline model has limits
Prior approaches have mostly converged on a pipeline model, wherein story
structure is generated as a precursor to discourse structure, mapping
individual story events to discourse elements. 
This model, however, unnecessarily limits narrative possibilities. %,
%which most prior work avoids by implicitly assuming a textual output
%medium.
%
% our approach 
We investigate a new generation approach that treats discourse as
primary, using {\em comic generation} as a testbed.  Our approach is based
on leading discourse theories for comics by McCloud (panel transitions) and
Cohn (narrative grammar). Rather than rearranging pre-existing panels, we
generate panel contents based on notions
% of {\em relatedness} 
supported by cognitive theories of visual language.
% results
We present a proof-of-concept generator with a wide range of abstract comic
output, a computational realization of McCloud's and Cohn's comics
theories, and a modular algorithm that affords the evaluation of visual
discourse theories.


