%================================================================
\section{Future Work}

% While our interpretation serves as a promising proof-of-concept for
% concretely interpreting theories of panel relatedness and visual grammar,
% we have identified a few limitations of our specific implementation
% choices.  Our choice to represent a panel as a frame and,
% independently, a set of VEs, means that VEs' relationship to the frame, or
% a VE's role in prior frames, is not available or manipulable.  By analogy
% with textual and verbal language, if a panel is analogous to a sentence,
% then we have grammar at the paragraph (narrative arc) level, but not at
% the sentence level.  Also, our choice to generate a transition sequence
% constrained by a grammar and {then} feed the transitions to a panel
% generator (itself a kind of pipeline model) means that the panel generator
% cannot reflect on the grammatical role of panels to guide its selection.
% In another iteration of the project, we would use linguistic theories to
% generate panel internals by assigning grammatical roles to VEs that pertain
% to their visual rendering (such as character, prop, or backdrop), then use
% those roles consistently across panel sequences. It may also be fruitful to
% reformulate transitions in terms of {\em edits} on previous panels that
% they are meant to be related to, rather than simply repeating VE sets.
% On the visual rendering level, edits may include scaling, zooming,
% backdrops, layers, and overlap among visual elements.  Theories of semantic
% scene composition, such as~\cite{zitnick2013bringing}, could yield a more
% principled basis for panel generation.

% The second avenue of future work is to extend the system's expressivity.
% Currently, our system cannot reason about certain key aspects of
% comics, such as captions and speech bubbles, as well as hierarchical structure,
% including two-dimensional panel arrays organized into rows and pages.  

Future directions for this project include expanding the set of visual
elements beyond abstract, geometric shapes (one candidate being the modular
XKCD sprite set (XXX ref)).
We would also like to incorporate generation into an interactive
framework, either for the purpose of interactive visual storytelling or
mixed-initiative comics design.
% The
% Storyteller\footnote{\url{http://www.storyteller-game.com/p/about-storyteller.html}}
% system in particular suggests an intriguing basis for comic-based play in
% which players select visual elements to populate a panel, and a reasoning
% engine finds a frame that connects it narratively to the panels on either
% side.  Such a system could also form the basis of a mixed-initiative comic
% design tool.

We
have several potential evaluation plans, each investigating distinct
hypotheses about our approach.
%
One candidate involves analyzing the style and variety of our
comic generator's output; i.e. our system's 
\emph{expressive range}~\cite{smith2010analyzing}. For this, and as
suggested by Smith and Whitehead, we would need to identify
appropriate metrics for describing the generated output, which ``should be
based on global properties \ldots and ideally should be emergent qualities
from the point of view of the generator;'' for example, the number and type
of transitions that are generated.
%
% in a large sample of generated comics. A cognitively-focused candidate
% metric is the average number of unique readings that an audience comes up
% with for generated comics. 
% Further, these metrics should be evaluated in
% the context of the discourse grammar's \emph{cyclomatic
% complexity}~\cite{mccabe1976complexity}, which in our case is low; such an
% analysis will yield insight into the representational power that the
% grammar has for generating narrative discourse, relative to the system's
% overall computational complexity.
%
Another candidate evaluation involves analyzing the level of comprehension
that our generated comics afford an audience. While there has been work in
understanding how people read into narratives involving abstract
shapes (e.g. the Heider-Simmel experiment~\cite{heider1944experimental}),
this evaluation would be more concerned with whether the discourse
categories (as discussed by Cohn) that guide the selection of transitions
are recognizable by an audience during comprehension.
Cohn~\cite{cohn2015narrative} discusses a methodology through which panel
discourse categories can be analytically identified; this analysis would
ask whether comic panel categories can be analytically identified by an
audience when they are intentionally selected by our generative system.




