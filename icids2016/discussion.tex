%================================================================
\section{Limitations and Future Work}

%There are three main avenues that we would like to explore to further develop
%this work: refining our discourse model, expanding the system's
%expressivity, and evaluating the generator.

While our interpretation serves as a promising proof-of-concept for
concretely interpreting theories of panel relatedness and visual grammar,
we have identified a few limitations of our specific implementation
choices.  Our choice to represent a panel as a frame and,
independently, a set of VEs, means that VEs' relationship to the frame, or
a VE's role in prior frames, is not available or manipulable.  By analogy
with textual and verbal language, if a panel is analogous to a sentence,
then we have grammar at the paragraph (narrative arc) level, but not at
the sentence level.  Also, our choice to generate a transition sequence
constrained by a grammar and {then} feed the transitions to a panel
generator (itself a kind of pipeline model) means that the panel generator
cannot reflect on the grammatical role of panels to guide its selection.
%
To address these above issues, future work will use linguistic theories to 
generate panel internals by assigning grammatical roles to VEs that 
pertain to their visual rendering (such as character, prop, or backdrop), 
then using those roles consistently across panel sequences. Further, we
will seek to reformulate transitions in terms of {\em edits} on previous 
panels they are related to, and incorporate theories of semantic scene 
composition (e.g. \cite{zitnick2013bringing}) to reason over the grammatical
role of panels in sequence.

%
%
%that they are meant to be related to, rather than simply repeating VE sets.
%Constrain the choice of frame as well.
%
%
%
%
%intend to incorporate theories of semantic scene composition 
%(e.g. \cite{zitnick2013bringing}), to expand.
%
%Further, we
%intend to 
%
%% In a second iteration of the project, we would make the following changes:
%\begin{itemize}
%\item use linguistic theories to generate panel internals by assigning
%grammatical roles to VEs that pertain to their visual rendering (such as
%character, prop, or backdrop), then use those roles consistently across
%panel sequences.
%\item On the visual rendering level, modify visual elements and frame
%descriptions to reason over notions such as scaling, zooming, backdrops,
%layers, and overlap among visual elements. We intend to study theories of
%semantic scene composition, such as~\cite{zitnick2013bringing}, as a more
%principled basis for panel generation.
%\item Reformulate transitions in terms of {\em edits} on previous panels
%that they are meant to be related to, rather than simply repeating VE sets.
%Constrain the choice of frame as well.
%\end{itemize}

%The second avenue of future work is to extend the system's expressivity.
%Currently, our system cannot reason about the following key aspects of
%comics:
%\begin{itemize}
%\item Text and images together, including captions and speech bubbles
%\item Hierarchically structured comics, such as two-dimensional panel
%arrays that need to have coherence and cohesion at the level of panel {\em
%rows}, and multi-page comics or graphic novels that need coherence and
%cohesion at the level of comic {\em pages}.
%\item Interactive comics. The
%Storyteller\footnote{\url{http://www.storyteller-game.com/p/about-storyteller.html}}
%system in particular suggests an intriguing basis for comic-based play in
%which players select visual elements to populate a panel, and a reasoning
%engine finds a frame that connects it narratively to the panels on either side.
%Such a system could also form the basis of a mixed-initiative comic design
%tool.
%\end{itemize}


Another avenue of future work is to empirically evaluate our system.
One candidate evaluation involves analyzing the style and variety of our
comic generator's output; i.e. our system's 
\emph{expressive range}~\cite{smith2010analyzing}. Some metrics based
on global properties that are emergent from the point of view of the
generator include\begin{inparaenum}[(a)]
	\item the number and type of transitions that are generated on average, and
	\item the average number of unique readings that an audience comes up
with for generated comics.
\end{inparaenum}
%
%These metrics should be evaluated in the context of the discourse grammar's 
%\emph{cyclomatic complexity}~\cite{mccabe1976complexity}, which in our case is low; 
%such an analysis will yield insight into the representational power that the
%grammar has for generating narrative discourse, relative to the system's
%overall computational complexity.
%
Another candidate evaluation involves analyzing the level of comprehension
that our generated comics afford an audience. While there has been work in
understanding how people read into narratives involving abstract
shapes (e.g. the Heider-Simmel experiment~\cite{heider1944experimental}),
this evaluation would be more concerned with whether the discourse
categories (as discussed by Cohn) that guide the selection of transitions
are recognizable by an audience during comprehension.
Cohn~\cite{cohn2015narrative} discusses a methodology through which panel
discourse categories can be analytically identified; this analysis would
ask whether comic panel categories can be analytically identified by an
audience when they are intentionally selected by our generative system.




