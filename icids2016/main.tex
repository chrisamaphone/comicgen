%"runningheads" enables:
%  - page number on page 2 onwards
%  - title/authors on even/odd pages
%This is good for other readers to enable proper archiving among other papers and pointing to content.
%Even if the title page states the title, when printed and stored in a folder, when blindly opening the folder, one could hit not the title page, but an arbitrary page. Therefore, it is good to have title printed on the pages, too.
\documentclass[runningheads,a4paper]{llncs}

%Even though `american`, `english` and `USenglish` are synonyms for babel package (according to https://tex.stackexchange.com/questions/12775/babel-english-american-usenglish), the llncs document class is prepared to avoid the overriding of certain names (such as "Abstract." -> "Abstract" or "Fig." -> "Figure") when using `english`, but not when using the other 2.
\usepackage[english]{babel}

%better font, similar to the default springer font
%cfr-lm is preferred over lmodern. Reasoning at http://tex.stackexchange.com/a/247543/9075
\usepackage[%
rm={oldstyle=false,proportional=true},%
sf={oldstyle=false,proportional=true},%
tt={oldstyle=false,proportional=true,variable=true},%
qt=false%
]{cfr-lm}
%
%if more space is needed, exchange cfr-lm by mathptmx
%\usepackage{mathptmx}

\usepackage{graphicx}

%extended enumerate, such as \begin{compactenum}
\usepackage{paralist}

%put figures inside a text
%\usepackage{picins}
%use
%\piccaptioninside
%\piccaption{...}
%\parpic[r]{\includegraphics ...}
%Text...

%Sorts the citations in the brackets
\usepackage{cite}

\usepackage[T1]{fontenc}

%for demonstration purposes only
\usepackage[math]{blindtext}

%for easy quotations: \enquote{text}
\usepackage{csquotes}

%enable margin kerning
\usepackage{microtype}

%tweak \url{...}
\usepackage{url}
%nicer // - solution by http://tex.stackexchange.com/a/98470/9075
\makeatletter
\def\Url@twoslashes{\mathchar`\/\@ifnextchar/{\kern-.2em}{}}
\g@addto@macro\UrlSpecials{\do\/{\Url@twoslashes}}
\makeatother
\urlstyle{same}
%improve wrapping of URLs - hint by http://tex.stackexchange.com/a/10419/9075
\makeatletter
\g@addto@macro{\UrlBreaks}{\UrlOrds}
\makeatother

%diagonal lines in a table - http://tex.stackexchange.com/questions/17745/diagonal-lines-in-table-cell
%slashbox is not available in texlive (due to licensing) and also gives bad results. This, we use diagbox
%\usepackage{diagbox}

%required for pdfcomment later
\usepackage{xcolor}

% new packages BEFORE hyperref
% See also http://tex.stackexchange.com/questions/1863/which-packages-should-be-loaded-after-hyperref-instead-of-before

%enable hyperref without colors and without bookmarks
\usepackage[
%pdfauthor={},
%pdfsubject={},
%pdftitle={},
%pdfkeywords={},
bookmarks=false,
breaklinks=true,
colorlinks=true,
linkcolor=black,
citecolor=black,
urlcolor=black,
%pdfstartpage=19,
pdfpagelayout=SinglePage,
pdfstartview=Fit
]{hyperref}
%enables correct jumping to figures when referencing
\usepackage[all]{hypcap}

%enable nice comments
\usepackage{pdfcomment}
\newcommand{\commentontext}[2]{\colorbox{yellow!60}{#1}\pdfcomment[color={0.234 0.867 0.211},hoffset=-6pt,voffset=10pt,opacity=0.5]{#2}}
\newcommand{\commentatside}[1]{\pdfcomment[color={0.045 0.278 0.643},icon=Note]{#1}}

%compatibality with TODO package
\newcommand{\todo}[1]{\commentatside{#1}}

%enable \cref{...} and \Cref{...} instead of \ref: Type of reference included in the link
\usepackage[capitalise,nameinlink]{cleveref}
%Nice formats for \cref
\crefname{section}{Sect.}{Sect.}
\Crefname{section}{Section}{Sections}

\usepackage{xspace}
%\newcommand{\eg}{e.\,g.\xspace}
%\newcommand{\ie}{i.\,e.\xspace}
\newcommand{\eg}{e.\,g.,\ }
\newcommand{\ie}{i.\,e.,\ }

%introduce \powerset - hint by http://matheplanet.com/matheplanet/nuke/html/viewtopic.php?topic=136492&post_id=997377
\DeclareFontFamily{U}{MnSymbolC}{}
\DeclareSymbolFont{MnSyC}{U}{MnSymbolC}{m}{n}
\DeclareFontShape{U}{MnSymbolC}{m}{n}{
    <-6>  MnSymbolC5
   <6-7>  MnSymbolC6
   <7-8>  MnSymbolC7
   <8-9>  MnSymbolC8
   <9-10> MnSymbolC9
  <10-12> MnSymbolC10
  <12->   MnSymbolC12%
}{}
\DeclareMathSymbol{\powerset}{\mathord}{MnSyC}{180}

% correct bad hyphenation here
\hyphenation{op-tical net-works semi-conduc-tor}


% ====================== CUSTOM COMMANDS 
%\DeclareMathAlphabet\mathbfcal{OMS}{cmsy}{b}{n}
\newcommand{\dref}[1]{Definition~\ref{#1}}
\newcommand{\eref}[1]{Equation~\ref{#1}}
\newcommand{\fref}[1]{Figure~\ref{#1}}
\newcommand{\tref}[1]{Table~\ref{#1}}
\newcommand{\sref}[1]{Section~\ref{#1}}
\newcommand{\aref}[1]{Algorithm~\ref{#1}}
\newcommand{\lpipe}{\rule[-0.4ex]{0.41pt}{2.3ex}\xspace}
\newcommand{\citeA}[1] {\citeauthor{#1}~(\citeyear{#1})}
\newcommand{\citeeg}[1] {(e.g. \citeauthor{#1}~\citeyear{#1})}

% Notes for draft-editing. I suggest you put your initials in the 
% first text env., and the note in the second text env.
% Syntax: \note
\def\note#1#2{\noindent {\color{red} {[\bf{#1}: {\it #2}]}}}

\begin{document}

%Works on MiKTeX only
%hint by http://goemonx.blogspot.de/2012/01/pdflatex-ligaturen-und-copynpaste.html
%also http://tex.stackexchange.com/questions/4397/make-ligatures-in-linux-libertine-copyable-and-searchable
%This allows a copy'n'paste of the text from the paper
\input glyphtounicode.tex
\pdfgentounicode=1

\title{Generating Abstract Comics}
%If Title is too long, use \titlerunning
%\titlerunning{Short Title}

%Single insitute
\author{Chris Martens \and Rogelio E. Cardona-Rivera}
% \author{Firstname Lastname \and Firstname Lastname}
%If there are too many authors, use \authorrunning
%\authorrunning{First Author et al.}
\institute{North Carolina State University\\
\{crmarten,recardon\}@ncsu.edu
}

%Multiple insitutes
%Currently disabled
%
% \iffalse
%Multiple institutes are typeset as follows:
% \author{Firstname Lastname\inst{1} \and Firstname Lastname\inst{2} }
%If there are too many authors, use \authorrunning
%\authorrunning{First Author et al.}

% \institute{
% Insitute 1\\
% \email{...}\and
% Insitute 2\\
% \email{...}
% }
% \fi
      
\maketitle

\begin{abstract}
  Comics use sequences of still, two-dimensional visual stimuli to tell
stories, and they are deeply entwined with humanity's creative history.
While there is prior work on computationally generating discourse
(conveyance of narrative) for textual stories, few attempts have been made
to target comics, which carry distinct challenges and affordances, as a
discourse language.  Standard pipeline-based approaches to narrative
generation, wherein a discourse is configured only after a fabula has been
fixed, do not map well onto the affordances of purely visual panel
sequences.

% XXX can we back up this claim more strongly?

% Prior efforts toward comic-authoring computer programs mainly
% conform to fixed panel contents which cannot be flexibly rearranged
% while retaining narrative coherence.

% solution we propose
We propose a {\em discourse-first} scheme for generating comics, including novel configurations
of panel contents, by combining a
bottom-up panel sequencer inspired by McCloud's taxonomy of panel
transitions and a top-down grammar of comic arcs devised by Cohn.
% results
Our approach yields a flexible algorithm for comic generation whose output
can be rendered with a number of visual palettes, for which we have built
one proof-of-concept example and observed a wide range of variability in
output. Our efforts represent a first step toward robust comic generation
and provide a platform for analyzing and evaluating the validity of visual
discourse theories.



\end{abstract}

\keywords{intelligent narrative technologies, comics, narrative generation}

%================================================================
\section{Introduction}

The computational generation of stories is an enterprise that can help us
precisely understand one of the most creative aspects of human
intelligence~\cite{boyd2009origin}: the ability to both generate stories about
possible and impossible worlds, and the ability to weave stories around our
daily life~\cite{herman2013storytelling}. In this paper, we begin to explore a
relatively unexplored sub-domain of computational narrative generation:~the
computational generation of comics. This targeted exploration is in the form of
a \emph{small-scale computational system}~\cite{montfort2012small} designed to
model what we believe are two key aspects of computationally creative narrative
generation:
%
\begin{inparaenum}[]
	\item a discourse-first orientation to the generation of narrative, and
	\item the importance of both hierarchical and sequential reasoning in 
		narrative generation.
\end{inparaenum}
%




\begin{itemize}
	\item Why are comics a great domain for computational creativity?
	\begin{itemize}
		\item Talk about how creative the discipline is
		\item Motivated by the exploration of computational creativity to novel domains
	\end{itemize}
	\item What are we trying to do?
	\item What is our approach?
	\item Why does discourse-driven help creativity in narrative generation?
		\begin{itemize}
			\item Creative potential is bigger in discourse-driven approach, principle 
			of least-commitment in the fabula - don't need to simulate a portion of 
			the narrative universe which is never revealed which may impose limitations
			on future narrative generation. \note{RCR}{Example!}
			
			\item Focusing on the telling may leave aspects of fabula unspecified, which 
			may broaden the interpretation of the story in the minds of story consumers. 
			\note{RCR}{Example!}
			\item Talk about the pipeline model of narrative generation (primarily
				simulation focused)
			\item We're exploring an alternative account - focus on the telling of the 
				story, let story consumers ``fill in the gaps''
		
			\begin{itemize}
				\item Gricean Maxims
				\item Closure principle Saraceni - Third aspect of ``relatedness'', 
				depends on not what is overtly explicit in the narrative's surface 
				code, but also on inference.  ``Closure'' comes from Gestalt principles.
				\item Gestalt principle 
			\end{itemize}	
		\end{itemize}

	
	\item Talk about Understanding Comics~\cite{mcCloud1993understanding}
	\item Talk about Visual Language of Comics~\cite{cohn2013visual}
\end{itemize}

We pursued a discourse-driven 



%================================================================
\section{On Generating Comics}

The computational generation of comics presents a novel challenge. Comics 
share structural similarity to written text~\cite{saraceni2016relatedness}: 
they are both made up of individual elements (sentences in text, frames in 
comics), delimited by special-purpose symbols (full stops in text, frame 
borders in comics), which can be easily identified, and which can contain a 
variable amount of information. However, unlike text, comics afford an additional \emph{pictorial} dimension through which to express information. To 
\citeA{cohn2013visual}, comics are one instance of the more broader 
\emph{theory of visual language}, which itself is used to communicate via
a palette of visual elements and their spatial relationships to one another,
e.g. their relative size, rotation, horizontal and vertical juxtaposition, 
and distance. While in general comics offer two authorship affordances 
(textual and visual language), in this work we are concerned only with 
the pictorial dimension.

One interesting and paradoxical aspect of purely-visual comics is how
constrained they for conveying narrative structure. The constraint is 
two-fold. Firstly, there is a practical constraint in that there are 
only so many visual elements that can be placed in a particular frame
before the ensemble becomes too littered to understand. Secondly, there
is an attentional constraint in that (even if we were to mediate
the first concern) a potential story consumer could be potentially
disengaged at the prospect of having to parse an overwhelming amount of
visual detail. There thus exists a tacit expectation on behalf of story
consumers that authors will make their contributions to the narrative arc
as brief and relevant as they need to be, in a manner parallel to the
expected conduct of people engaged in cooperative conversation as outlined 
by \citeA{grice1975logic}.

Thus, for the enterprise of computational narratology, it seems prudent to
understand and computationally model the constraints of discourse, since 
it is the primary point of contact with the narrative artifact. While we 
acknowledge that story structure is important for 
comprehension~\cite{graesser2002how}, we feel that the story exists primarily
in the head of the narrative consumer, and that story structure is recovered 
by the consumer insofar it is afforded by the discourse structure. In particular
for the case of comics, \citeA{saraceni2016relatedness} argues that a comic's
discursive structure ought to afford readers a sense of \emph{relatedness}, 
between comic elements.

%	 	Indeed, \citeA{saraceni2016relatedness} argues that comic structure 
%		ought to demonstrate \emph{relatedness}, which spans from the structural
%		aspects of discourse to the cognitive aspects of discourse (which turns
%		out to be the structural aspect of stories, needed for sense-making).

\note{RCR}{Discuss the three elements of Saraceni's account, including the closure principle (third aspect of \emph{relatedness}, which depends on not what is overtly explicit in the narrative's surface code, but also on inference.  Borrowed from visual perception and Gestalt principles.}


\note{RCR}{I think a figure that illustrates the spectrum of structural-relatedness to cognitive-relatedness would be good here.}

In our work we sought to develop a small scale computational model, and thus focused primarily on modeling discourse structure which lies on the structural side of the spectrum. However, our discourse model includes a minimal model of story, which is needed in order to account for some elements of the cognitive side of the spectrum.  We developed two computational models of discourse structure: one based on~\citeauthor{mcCloud1993understanding}'s~(\citeyear{mcCloud1993understanding}) account of \emph{transition types}, and the other based on \citeauthor{cohn2013visual}'s~(\citeyear{cohn2013visual}) \emph{theory of visual language}.


% Human brains' ability to fill in gaps is also why comics are simpler than
% animation in this respect: animations are expected to provide continuous
% motion between frames, whereas two comic frames need only be plausibly
% connected by some narrative justification. And that's where transition types
% come in: when you exclude non sequiturs, they constrain the space of next
% panels to ones that "make sense."



\section{System Description}

Our approach to generating visual narratives begins as a linear
process that selects next comic panels based on the contents of previous
panels, choosing randomly among indistinguishably-valid choices.
The concepts we represent formally are {\em transitions}, {\em frames}, and
{\em visual elements}, which we define below.

% XXX how to make this a heading that looks different from a section
% heading?
% \subsection{Visual elements, frames, and transitions}

A {\bf visual element (VE)} is a unique identifier from an infinite set,
each of which is possible to map to a distinct visual representation.
We do not explicitly tag visual elements as specifically characters, props,
or scenery, making the representation agnostic to which of these narrative
interpretations will apply. In the visual rendering of our comics, we
represent VEs as random combinations of shape, color, and size, supplying
additional inputs to the human cognitive processes that may interpret these
elements' narrative role.

A {\bf frame} is a panel template; at the abstract generation level, it
includes an identifier or set of tags and a minimum number of required
visual elements. The reason a frame specifies a {\em minimum} number of VEs
is to allow for augmentation of the frame with pre-existing elements: for
example, the {\em monologue} frame requires at least one visual element,
indicating a single, central focal point, but other visual elements may be
included as bystanding characters or scenery elements.
At the rendering level, a frame includes instructions for where in the
panel to place supplied visual elements.
A {\bf panel} is a frame instantiated by specific visual elements.

% Modifier: visual details overlaid on frames and VEs to add semantic
% coherence to the comic, such as floating emotes, facial expressions, motion
% lines, word balloons, and other text.

Finally, a {\bf transition} is a specification for how a panel should be
formed as the next panel in a sequence, which we describe formally below.

Transition types were first described by
McCloud~\cite{mcCloud1993understanding} % XXX as a means of analyzing
comics. He gave an account of transitions including {\em moment-to-moment},
{\em subject-to-subject}, and {\em aspect-to-aspect}, referring to changes
in temporal state, focal subjects, and spatial point-of-view. As Cohn (XXX
cite ch 4 of visual lang of comics) points out, these transition types are
highly contextual; they presume the reader has a semantic model of the
``story world'' in which the comic takes place. For the sake of
computational generation, we derive a more {\em syntactic} notion of
transition defined purely in terms of frames and (abstract) visual
elements. So, for example, while McCloud could refer to an action-to-action
transition as one where a character is depicted carrying out two distinct
actions, we have no notion of {\em character} and {\em action}, so instead
must refer to which visual elements appear and in which frame. The
rendering of a frame itself may position VEs in such a way that a reader
would read certain actions or meaning into it; however, this kind of reader
interpretation is not modeled to inform generation.

\subsection{Formal Transition Types}

We introduce six formal transition types: {\bf moment}, {\bf add}, {\bf
subtract}, {\bf meanwhile}, and {\bf rendez-vous}, each of which specifies
how a next panel should be constructed given the prior sequence.

\begin{itemize}
\item {\bf Moment} transitions retain the same set of VEs as the previous panel, 
changing only the frame.

\item {\bf Add} transitions introduce a VE that didn't appear in the
previous panel, but might have appeared earlier (or might be completely
new). A new frame may be selected.

\item {\bf Subtract} transitions remove a VE from the previous panel and
potentially choose a new frame.

\item {\bf Meanwhile} transitions select a new frame and show {\em only}
VEs that did not appear in the previous panel, potentially generating new
VEs.

\item {\bf Rendez-vous} transitions select a random subset of
previously-appearing VEs (from anywhere in the sequence) and selects a new
frame to accommodate them.
\end{itemize}

\subsection{Implementation}

(XXX)

\section{Generator Output}

(XXX)

\subsection{Example}

XXX flesh out:
\begin{itemize}
\item Show the interface
\item Show an example of output, both in abstract and rendered form
\end{itemize}

\begin{figure}
\includegraphics[width=0.5\textwidth]{comicgen-output-1.png}
\end{figure}

\subsection{Constraining generation with Cohn grammars}



\subsection{Example of constrained output}






%================================================================
\section{Related Work}

As discussed in the Introduction, the pipeline model of narrative generation 
has been the dominant paradigm to narrative generation. In this section we
review some exemplars of that model, with special focus on systems that have
been covered in the computational creativity community.

\citeA{guerrero2014social} developed a nuanced computational model of social
norms to drive the interaction of characters in the simulation of the story
world. Their work defers the development of the main plot to
MEXICA~\cite{perez2001mexica}, a computational implementation of a
cognitively-oriented account of writing. However MEXICA itself is primarily a
story-level reasoner, since it leaves unspecified how the story structures that
it generates via computational \emph{engagement} and \emph{reflection} are 
realized into narrative text. 

While MEXICA itself follows the pipeline model of narrative generation, its
engagement--reflection (E--R) model of authorship is relevant to our work. 
The E--R cycle represents a \emph{tandem-process model}, which is similar to our 
account of discourse reasoning. In MEXICA, the plot
elaboration component (the \emph{engagement} phase) is responsible for 
constructing an initial story framework, which is refined by a critic (the
\emph{reflection} phase). In our work, the discourse elaboration component
(the local reasoner) is responsible for constructing an initial discourse
structure, which is refined by a critic (the global reasoner).  Further, 
the E--R cycle is a \emph{cognitively-oriented} narrative generation process;
\citeauthor{perez2001mexica} leveraged information on how humans cognitively 
engage with the narrative authorship process in order to inform their system 
design. In our work, we too took a cognitive orientation by looking at how 
humans parse comic discourse structure to inform the design of our comic 
discourse generator.

\citeA{montfort2013slant} developed a blackboard architecture called Slant for
story generation that integrates several different sub-components systems to
generate a story. While the system's architecture is primarily dedicated to the
specification and refinement of rules to generate plot structure, Slant does
include a sub-component called Verso, which reasons over narrative discourse as
a way to further constrain the narrative plot. In particular, Verso detects
aspects of the verbs used during the generation of plot structure, and
determines the in-progress story's match to a specific genre.\footnote{Verso's
operationalization of genre differs from the literary sense of the term, but a
full discussion of this is beyond the scope of our work.} Once a specific genre
has been identified, Verso poses additional constraints to the plot generator
via the Slant blackboard. Slant is thus not strictly a pipeline model
architecture, but unfortunately the constraints identified during discourse
reasoning cannot themselves inform further discourse reasoning. In our approach,
we hope to identify discourse-driven narrative generation that informs or
constrains both the generation of the underlying plot structure, as well as the
further generation of narrative discourse.

Most relevant to the work we pursue here is the work by
\citeA{perezyperez2012illustrating}, who developed a visual illustrator to their
MEXICA system. They sought to verify the degree to which their 3-panel comic 
generator elicited in readers the same sense of story as a textual realization 
of the same MEXICA-generated plot. While this system still follows the pipeline
model of narrative generation, we see their work as complementary: they developed 
an experiment methodology through which it is possible to empirically assess if
their palette of designed visual elements denote story concepts as intended. 
Future work in discourse-driven comic generation will have to address this
point going forward, and \citeauthor{perezyperez2012illustrating} provide a step
toward understanding the gap between story concepts and the computational symbols
meant to encode them. A potential improvement to their system that the authors 
identify as most important was: ``to provide the Visual Narrator with mechanisms
that allow more freedom during the composition 
process''~\cite{perezyperez2012illustrating}. Our work here aims to provide just 
that.


%================================================================
\section{Future Work}

There are three main avenues that we would like to explore to further develop
this work.

The first avenue is with respect to a refinement of our system's discourse 
model. We would like to refine our model in several key ways, including\ldots
Reformulating the implementation:
\begin{itemize}
\item Connecting panel internals to narrative structure: roles for visual
elements
\item Exchanging visual element palettes along a spectrum of abstraction
\item Reformulating transitions in terms of ``edits'' on previous panels
rather than simply their frame/VE sets -- which VE gets assigned to which
frame position is lost information, for example
\end{itemize}

The second avenue is with respect to the system's expressivity. Currently, our
system cannot reason about the following key aspects\ldots
\begin{itemize}
\item Text and images together
\item More hierarchical structure -- sequences of comic ``pages'';
larger-scale stories
\end{itemize}

The third avenue is with respect to the empirical evaluation of this work.
There are several ways that we could evaluate our system, and each evaluation
would lend different strength to our approach. One candidate evaluation involves
analyzing our system's \emph{expressive range}~\cite{smith2010analyzing}.

Another candidate evaluation involves analyzing the level of comprehension
that our generated comics afford an audience. While there has been work in
understanding how people read into narratives involving abstract
shapes~\citeeg{heider1944experimental}, this evaluation would be more 
concerned with whether the discourse categories (as discussed by Cohn) that
guide the selection of transitions are recognizable by an audience during
comprehension. \citeA{cohn2015narrative} discusses a methodology through 
which panel discourse categories can be analytically identified; we are 
interested in whether comic panel categories can be analytically identified 
when they are intentionally selected by our generative system.

\note{RCR}{Other evaluation-focused work?}








\section{Conclusion}

In this work we have presented a discourse-driven approach to narrative 
generation in contrast to most existing work within the computational 
creativity community, which has primarily followed a pipelined approach.
We initially designed our system to pay attention to mostly textual
factors in comic discourse: the repetition of comic actants across the
narrative provides a minimal cohesive backbone on which to pin comic 
understanding.
%
However, as discussed, this form of generation could generate non-sensical
output (e.g. ending comics with a \emph{meanwhile} discourse transition). 
We therefore appealed to more cognitively-oriented factors via the theory
of visual grammar, which helped structure the output in a way that enables
other senses of relatedness to contribute to the output's coherence.
%
Thus, through our small-scale system, we have begun to explore the scale and
limits of human story sense-making faculties, as well as how they come to
bear on narrative generation systems: in our case, through both local and 
global procedures, which inform cohesion and coherence, respectively. 
%
Our algorithms and implementation offer a promising starting point for the
computational investigation of discourse-driven narrative.

More broadly, our work highlights the importance of looking to human
cognition as a point of departure for the design of narrative generators.
Other scholars~(e.g.
\citeauthor{gervas2009computational}~\citeyear{gervas2009computational}, 
\citeauthor{szilas2010requirements}~\citeyear{szilas2010requirements})
have argued the same point; our system provides a computational system
that demonstrates it. Concretely, the reason for this is that humans 
bring significant cognitive faculties to bear on the process of narrative 
comprehension~\cite{herman2013storytelling}. 
An instance of this narrative intelligence is our unique ability to fill 
in the blanks in the liminal spaces of discourse, which (at least) relies on  
our focalized perspectives into the story world~\cite{genette1983narrative}.
As our generated comics show, our narrative sense-making abilities allow us 
to intuit and impose narrative structure on the sequence of depicted images, 
due to how we fill in the blanks left unspecified in our comics. Therefore,
this mental process has a significant role in our appreciation of the 
narrative artifact, and should have an equally significant role in the
generation of it.


{
\bibliographystyle{splncs03}
\bibliography{main}
}

\end{document}
