%================================================================
\section{Related Work}

%Guerrero and P\'erez y P\'erez~\cite{guerrero2014social} developed a
%nuanced computational model of social norms to drive the interaction of
%characters in the simulation of the story world. Their work defers the
%development of the main plot to MEXICA~\cite{perez2001mexica}, a
%computational implementation of a cognitively-oriented account of writing.
%However MEXICA itself is primarily a story-level reasoner, since it leaves
%unspecified how the story structures that it generates via computational
%\emph{engagement} and \emph{reflection} are realized into narrative text. 

%While MEXICA itself follows the pipeline model of narrative generation, its
%engagement--reflection (E--R) model of authorship is relevant to our work. 
%The E--R cycle represents a \emph{tandem-process model}, which is similar to our 
%account of discourse reasoning. In MEXICA, the plot
%elaboration component (the \emph{engagement} phase) is responsible for 
%constructing an initial story framework, which is refined by a critic (the
%\emph{reflection} phase). In our work, the discourse elaboration component
%(the local reasoner) is responsible for constructing an initial discourse
%structure, which is refined by a critic (the global reasoner).  Further, 
%the E--R cycle is a \emph{cognitively-oriented} narrative generation process;
%P\'erez y P\'erez leveraged information on how humans cognitively 
%engage with the narrative authorship process in order to inform their system 
%design. In our work, we too took a cognitive orientation by looking at how 
%humans parse comic discourse structure to inform the design of our comic 
%discourse generator.

Montfort et al.~\cite{montfort2013slant} developed a blackboard
architecture called Slant for textual story generation, which primarily
specifies and refines plot structure. However, Slant
% that integrates several different sub-components systems to generate a story. 
includes a sub-component called Verso, which reasons over narrative 
discourse as a way to further constrain the narrative plot. Verso 
detects aspects of the verbs used during the generation of plot structure, 
and determines the in-progress story's match to a specific genre.
Slant is thus not strictly a pipeline model architecture, but the constraints 
identified during discourse reasoning cannot themselves inform further 
discourse reasoning. In our approach, discourse reasoning constrains plot structure
generation, and can potentially inform further narrative discourse generation.

P\'erez y P\'erez et al.~\cite{perezyperez2012illustrating} developed a 
visual illustrator to the MEXICA system~\cite{perez2001mexica}, and verified
%
the degree to which their 3-panel comic generator 
elicited in readers the same sense of story as a textual realization of 
the same MEXICA-generated plot. 
%
While this system still follows the pipeline model of narrative 
generation, we see their work as complementary: they developed an experiment 
methodology through which it is possible to empirically assess if
their palette of designed visual elements denote story concepts as intended. 
Future work in discourse-driven comic generation will have to address this
point going forward, and P\'erez y P\'erez et al. provide a step
toward understanding the gap between story concepts and the pictorial symbols
meant to encode them. 

%A potential improvement to their system that the authors 
%identify as most important was: ``to provide the Visual Narrator with mechanisms
%that allow more freedom during the composition 
%process''~\cite{perezyperez2012illustrating}. Our work here aims to provide just 
%that.



