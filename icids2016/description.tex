\section{System Description}

Our approach to generating visual narratives begins as a linear
process that selects next comic panels based on the contents of previous
panels, choosing randomly among indistinguishably-valid choices.
The concepts we represent formally are {\em transitions}, {\em frames}, and
{\em visual elements}, which we define below. There are two levels on which
to make sense of these terms: the symbolic level, i.e. the intermediate,
human-readable program datastructures representing a comic, and the
rendered level, designed to be consumed by human visual perception.

A {\bf visual element (VE)} is a unique identifier from an infinite set,
each of which is possible to map to a distinct visual representation.
We do not explicitly tag visual elements with their roles in the narrative,
such as characters, props, or scenery, making the symbolic representation
agnostic to which of these narrative interpretations will apply. In the
visual rendering, of course, our representation choices will influence
readers' interpretation of VEs' narrative roles.

A {\bf frame} is a panel template; at the symbolic level, it
includes an identifier or set of tags and a minimum number of required
visual elements. The reason a frame specifies a {\em minimum} number of VEs
is to allow for augmentation of the frame with pre-existing elements: for
example, the {\em monologue} frame requires at least one visual element,
indicating a single, central focal point, but other visual elements may be
included as bystanding characters or scenery elements.
At the rendering level, a frame includes instructions for where in the
panel to place supplied visual elements.
A {\bf panel} is a frame instantiated by specific visual elements.

% Modifier: visual details overlaid on frames and VEs to add semantic
% coherence to the comic, such as floating emotes, facial expressions, motion
% lines, word balloons, and other text.

Finally, a {\bf transition} is a specification for how a panel should be
formed as the next panel in a sequence.
%
% Now said in prev section
% Transition types were first described by McCloud~\cite{mcCloud1993understanding} 
% as a means of analyzing
% comics. He gave an account of transitions including {\em moment-to-moment},
% {\em subject-to-subject}, and {\em aspect-to-aspect}, referring to changes
% in temporal state, focal subjects, and spatial point-of-view. As 
% Cohn~\cite{cohn2013visual} (Chapter 4) points out, these transition types are
% highly contextual; they presume the audience has a semantic model of the
% story world in which the comic takes place. 
We took inspiration from McCloud
transitions~\cite{mcCloud1993understanding}, developing a more
syntactic notion  defined purely in terms of frames and (abstract) visual
elements, for which Saraceni's theory of
relatedness~\cite{saraceni2016relatedness} could be applied. For example,
while McCloud could refer to an action-to-action transition as one where a
character is depicted carrying out two distinct actions, we have no notion
of {\em character} and {\em action} (these being semantic and contextual
categories), so instead must refer to which visual elements appear, where
they have appeared previously, and what their spatial relationships might
be (potential frames). The rendering of a frame itself may position VEs in
such a way that an audience would read certain actions or meaning into it;
however, this kind of audience interpretation is not modeled to inform
generation.

\subsection{Formal Transition Types}

We introduce six formal transition types: {\em moment}, {\em add}, {\em
subtract}, {\em meanwhile}, and {\em rendez-vous}, each of which specifies
how a next panel should be constructed given the prior sequence.

\begin{itemize}
\item {\bf Moment} transitions retain the same set of VEs as the previous panel, 
changing only the frame.

\item {\bf Add} transitions introduce a VE that didn't appear in the
previous panel, but might have appeared earlier (or might be completely
new). A new frame may be selected.

\item {\bf Subtract} transitions remove a VE from the previous panel and
potentially choose a new frame.

\item {\bf Meanwhile} transitions select a new frame and show {\em only}
VEs that did not appear in the previous panel, potentially generating new
VEs.

\item {\bf Rendez-vous} transitions select a random subset of
previously-appearing VEs (from anywhere in the sequence) and selects a new
frame to accommodate them.
\end{itemize}

\subsection{Implementation}

Our generator accepts as inputs length constraints (minimum and maximum)
and a number of VEs to start with in the first panel. Its output is a
sequence of panels (frame names and VE sets) together with a record of the
transitions that connect them.

The generation algorithm is:
\begin{enumerate}
\item Generate transition sequence by choosing transitions uniformly at
random, constrained by supplied minimum and maximum length.
\item Generate unique identifiers matching the number of specified starting
VEs.
\item Feed transition sequence and starting VEs to the panel sequencer,
which selects a next frame and VE set for each new panel based on each
transition type's definition (described above). Generate new VEs when
necessary, updating the running pool of previously-used VEs at each
iteration.
\end{enumerate}

% In addition to our OCaml implementation
% XXX unblind
% , which can be found on GitHub,\footnote{\url{http://www.github.com/chrisamaphone/comicgen}} we
% have implemented a web front-end in JavaScript.\footnote{
%\url{http://www.cs.cmu.edu/~cmartens/comicgen}} 
We implemented this algorithm in OCaml and additionally implemented a
front-end, a web-based renderer (not linked here for anonymous review). The
renderer assigns each frame type to a set of coordinates given by
percentage of the vertical and horizontal panel size, and then renders
panels by placing visual elements at those coordinates. Visual elements are
represented by randomly generated combinations of size, shape (circle or
rectangle), and color.  An example of the generator's output can be seen in
Figure~\ref{fig:out1}. 

\begin{figure}
\includegraphics[width=\columnwidth]{comicgen-unconstrained-ok.png}
\caption{Example of generator output.
While the narrative here is ambigious, we read several things into it:
the repetition of the grey (largest) rectangle in every frame suggests it
as a focal point, and the sudden appearance of the pink (smallest)
rectangle suggests an interloper removing the grey rectangle from its
initial context (established by the blue rectangle and yellow circle).
Together with the names of the frames (reported in symbolic form above the
comic), we can read the sequence as follows: the grey rectangle whispers to
the blue rectangle, then is carried off by a pink rectangle, who whispers
to the grey rectangle and then aids the grey rectangle.
}
\label{fig:out1}
\end{figure}

\subsection{Constraining Generation with Cohn Grammars}
Generating random transition sequences may result in nonsensical output,
such as ending a comic with a \emph{meanwhile} frame in which completely new
visual elements are introduced at the end of the comic, but not connected to 
previous elements; see Figure~\ref{fig:outbad} for an example. 

\begin{figure}
\includegraphics[width=\columnwidth]{comicgen-underconstrained-2.png}
\caption{Example of underconstrained output. The final panel does not
maintain relatedness to the preceding sequence.}
\label{fig:outbad}
\end{figure}

In an attempt to understand the global structure of comic panel sequences,
Cohn~\cite{cohn2016visual} and his colleagues investigate the {\em linguistic structure} of
visual narratives. They claim that understandable comics follow a grammar
that organizes its global structure. Instead of transition types, Cohn's
grammar of comics consists of grammatical categories (analogous to nouns,
verbs, and so on) indicating the role that each panel plays in the
narrative. These categories are {\bf establisher}, {\bf initial}, {\bf
prolongation}, {\bf peak}, and {\bf release}, which allow the formation of
standard narrative patterns including the Western dramatic arc of 
\hbox{\em initial -- peak -- release}. Formally, Cohn gives the following grammar 
as a general template for comic ``sentences,'' or well-formed arcs:

{\it (Establisher) -- (Initial (Prolongation)) -- Peak -- (Release)}

\noindent Symbols in parentheses are optional. In our expression of this grammar (and
in several of Cohn's examples), we also assume that prolongations may occur
arbitrarily many times in sequence.

Grossman built a generator based purely on Cohn's arc grammar,\footnote{
  \url{http://www.suzigrossman.com/fineart/conceptual/Sunday_Comics_Scrambler}
}
picking hand-annotated panels for each of an {\em initial},
{\em peak}, and {\em release} slot in the comic. However, this generation
scheme does not manipulate the internal structure of the comic panels,
allowing for less variability in the output than our scheme with visual
elements and frames. Additionally, codifying the syntactic structure of
individual panels allows us to characterize {\em relatedness} between
panels as described by Saraceni~\cite{saraceni2016relatedness}. In our second
iteration of the generator, we combine two approaches to discourse, using
{\em global} Cohn grammars to guide the {\em local} selection of
syntactically-defined transitions.

% What I want to say is: we don't just benefit from Cohn's work;
% he benefits from ours, because we're operationalizing semantic panel
% roles in terms of the panel's syntactic relationship to prior panels.
% I.e. in the linguistic metaphor, we're rejecting "panels as words" and
% instead treating "panels as utterances" and their constituent VEs as
% words.
% Maybe it's worth suggesting that panels themselves could obey a grammar,
% rather than haphazardly plugging VEs into different frame positions?

In particular, we enumerate every possible category bigram in Cohn's grammar,
such as {\em initial to prolongation}, {\em prolongation to peak}, and so
on, and describe sets of transition types that could plausibly model the
relationship. This mapping is given below:
% in \fref{figure:transitions}.

% \begin{figure}

% \begin{table}

{\scriptsize
\begin{tabular}{lll}
Establisher & Initial & \{Moment, Subtract, Add, RendezVous\} \\
   Establisher & Prolongation & \{Moment, Subtract, Add\} \\
   Establisher & Peak & \{Add, Meanwhile\} \\
   Initial & Prolongation & \{Moment, Subtract, Add\} \\
   Prolongation & Prolongation & \{Moment, Subtract, Add\} \\
   Prolongation & Peak & \{Subtract, Add, RendezVous\} \\
   Initial & Peak & \{Subtract, Add, Meanwhile, RendezVous\} \\
   Peak & Release & \{Subtract, Add, RendezVous\}
\end{tabular}
}
% \end{table}

% \caption{Mapping from grammatical category bigrams to possible sequential transition types.}
% \label{figure:transitions}
% \end{figure}

This particular mapping is guided by our intuition rather than any kind of
systematic symbolic reasoning---we aim to rule out obvious-seeming
syntactic errors, e.g. a \emph{meanwhile} transition at the end of an arc, but
other constraints are not so easily expressed. For instance, perhaps a
\emph{prolongation} should be realized as a repetition of the previous
transition, but this information is not available in the bigram model. In
future work, we would like to refine the theoretical grounding of the
relationship between transitions and grammatical categories.

With this mapping established, we randomly generate an instance of the
arc grammar and populate it with an appropriate set of transitions, after
which point we simply hook the transition sequence up to the same panel
selector from before.

Examples of the constrained generator's output can be found in
Figures~\ref{fig:outgood} and~\ref{fig:redgreen}.


\begin{figure}
\includegraphics[width=\columnwidth]{output-constrained-canonical.png}
\caption{Example of grammatically-constrained output.
This example shows a common pattern in grammatically-constrained output,
introducing a new visual element with a Meanwhile transition for the peak,
then releasing with a Rendez-vous.
}
\label{fig:outgood}
\end{figure}

\begin{figure}
\includegraphics[width=\columnwidth]{comicgen-output-4.png}
\caption{Example of grammatically-constrained output
illustrating a longer sequence with just three visual elements. Potential
narrative readings include the green circle throwing the smaller red
circle; the larger red circle can be seen as a different entity or as the
extension of the smaller one. The peak of this arc is the second-to-last
panel.
}
\label{fig:redgreen}
\end{figure}



