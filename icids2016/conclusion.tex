\section{Conclusion}

In this work we have presented a discourse-driven approach to narrative 
generation in contrast to most existing work, which has primarily 
followed a pipelined approach. We initially designed our system to pay 
attention to mostly textual factors in comic discourse: the repetition 
of comic actants across the narrative provides a minimal cohesive backbone 
on which to pin comic understanding.
%
However, as discussed, this form of generation could generate non-sensical
output (e.g. ending comics with a \emph{meanwhile} discourse transition). 
We therefore appealed to more cognitively-oriented factors via the theory
of visual grammar, which helped structure the output in a way that enables
other senses of relatedness to contribute to the output's coherence.
%
Thus, through our small-scale system, we have begun to explore the scale and
limits of human story sense-making faculties, as well as how they come to
bear on narrative generation systems: in our case, through both local and 
global procedures, which inform cohesion and coherence, respectively. 
%
Our algorithms and implementation offer a promising starting point for the
computational investigation of discourse-driven narrative.

More broadly, our work highlights the importance of looking to human
cognition as a point of departure for the design of narrative generators.
Other scholars~(e.g. Gervas~\cite{gervas2009computational} and
Szilas~\cite{szilas2010requirements}) have argued the same point; our
system provides a computational system that demonstrates it. Concretely,
the reason for this is that humans bring significant cognitive faculties to
bear on the process of narrative
comprehension~\cite{herman2013storytelling}.  An instance of this narrative
intelligence is our unique ability to fill in the blanks in the liminal
spaces of discourse, which (at least) relies on  our focalized perspectives
into the story world~\cite{genette1983narrative}.  As our generated comics
show, our narrative sense-making abilities allow us to intuit and impose
narrative structure on the sequence of depicted images, due to how we fill
in the blanks left unspecified in our comics. Therefore, this mental
process has a significant role in our appreciation of the narrative
artifact, and should have an equally significant role in the generation of
it.
