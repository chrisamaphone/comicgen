%================================================================
\section{On Generating Comics}

Skilled authors convey their stories with knowledge of how information is
likely to be processed by an audience.  Readers learn to optimize their
consumption of relevant information~\cite{pirolli2007information}, and work
to construct inferences~\cite{magliano2016filling} about story content in
the liminal spaces of discourse (in between sentences in text, panels in
comics, scenes in film); inferences for story content are constructed when
they are \emph{needed} for comprehension, and \emph{enabled} by what has
been narrated thus far~\cite{myers1987degree}. All told, the dynamic
between story authors and audiences parallels the dynamics of people
engaged in cooperative conversation as outlined by the philosopher of
language Grice~\cite{grice1975logic}: the storyteller, as the active contributor
to the ongoing communicative context, is expected to make her contributions
to the discourse based on what is relevant to her narrative intent. 
As Murray~\cite{murray2011why} states:
%
\begin{quote} 
	In a mature medium nothing happens, nothing is brought on stage (or screen 
	or comic book panel or described in prose) that does not in some way further 
	the action. Whatever the viewer is invited to direct attention to is
	something that further defines the role ([of a] character) or the function 
	(dramatic beat). 
	\end{quote}
%
These expectations give rise to narrative devices such as \emph{Chekhov's
gun}, wherein narrative elements are introduced because they are relevant, 
and they ultimately demonstrate their relevance at some point in the story.
Narrative authors can at the same time flout this expectation of 
cooperativity in service of a counterpart narrative device, the 
\emph{red herring}, wherein a story element is introduced and which
ultimately has no relevance to the unfolding story.

Thus, for the enterprise of computational narratology, it seems prudent to
encode the constraints and effects of narrative discourse, since (as the
primary point of contact with the narrative artifact) narrative discourse
carries with it expectations and conventions that ultimately affect how
story consumers understand the narrative. While we acknowledge that
coherent story structure is important for
comprehension~\cite{graesser2002how}, the audience recovers that structure
only insofar as it is afforded by the discourse structure. 

%\note{CRM}{Somewhere in the below commentary, it would make sense to cite
%~\cite{heider1944experimental}; I'm not sure where.}

{\em Purely visual comics}, or sequences of visual imagery arranged in
panels, present an excellent avenue along which to study discourse
theories computationally. 
The same principles apply: comprehensible comics
lack visual clutter, and differences across the {\em gutters} (gaps between
panels) are designed to be filled in by an audience's inference. 
These principles, as well as notions of brevity, relatedness, and other
principles of cooperative narration, manifest in terms of discrete
particles that are easily recognized and generated by computer programs. 

Comics are structurally similar to written
text~\cite{saraceni2016relatedness}: they are both made up of individual
elements (sentences in text, panels in comics), delimited by special-purpose
symbols (full stops in text, panel borders in comics), which can be easily
identified, and which can contain a variable amount of information. However,
unlike text, comics afford an additional \emph{pictorial} dimension through
which to express information via a palette of visual elements and their spatial
relationships to one another, e.g. their relative size, rotation, horizontal and
vertical juxtaposition, and distance. While in general comics offer two
dimensions of authorship affordances (textual and visual language), in this work
we are concerned only with the pictorial dimension.
%
\begin{figure}
	\includegraphics[width=\columnwidth]{relatedness.png}
	\caption{
		The spectrum of \emph{relatedness} as discussed by
                Saraceni~\cite{saraceni2016relatedness}. Relatedness indicates how 
		comic panels are connected or associated in the minds of 
		readers, spanning from textual factors to cognitive factors. 
		Along that spectrum, there are three  distinguished 
		categories of relatedness: \emph{repetition}, 
		\emph{collocation}, and \emph{closure}, which have
		demonstrably different effects on the construction of
		narrative mental models.
		}
	\label{figure:relatedness}
\end{figure}
%
Saraceni~\cite{saraceni2016relatedness} describes three notions of
\emph{relatedness} between comic elements.
Relatedness, a property of a comic that indicates how its panels are
connected or associated, depends on a comic's \emph{cohesion} -- the
lexico-grammatical features that tie panels together -- and \emph{coherence} --
the audience's perception of how individual panels contribute to her mental model
of the unfolding events. Relatedness emerges from a spectrum of \emph{textual}\footnote{\emph{Textual} here does not mean the use of actual text, but rather is a shorthand for \emph{surface code}~\cite{zwaan1998situation}.}
factors to \emph{cognitive} factors, illustrated in \fref{figure:relatedness}.
%
Saraceni distinguishes three categories of relatedness.
Closer to the textual end of the spectrum is the \emph{repetition} of visual
elements across panels. Beyond repetition is \emph{collocation}, which refers
to an audience's expectation that related visual elements will appear given the
ones that have been perceived. Closer to the cognitive end of the spectrum is
the \emph{closure} over comic elements, which refers to the way our minds 
complete narrative material given to us. Closure is terminologically borrowed 
from the field of visual cognition, but is intended as the mental process 
of inference that occurs as part of an audience's 
\emph{search for meaning}~\cite{gerrig1994readers}.

%
\begin{figure}[t]
	\includegraphics[width=\columnwidth]{xkcd-to_taste.png}
	\caption{
		Strip \#1639 of {\em XKCD}, {\small\copyright} Randall Munroe. This comic
		depends on three aspects of relatedness as described by 
                Saraceni~\cite{saraceni2016relatedness}, and as illustrated in 
		\fref{figure:relatedness}.
	}
\label{fig:xkcd}
\end{figure}
In Figure~\ref{fig:xkcd}, we see a comic that depends on the three aforementioned 
aspects of relatedness: first, repetition of the stove and pot is used to maintain
cohesion across panels. Second, the punchline of the comic depends on
collocation in the sense that we expect ``sugar'' to come in small
measurements, based on non-grammatical domain knowledge.  Finally, the
comic depends on closure in that we expect the audience to infer several
things: that before the start of the comic, the character had been
following a recipe; that the character went to retrieve the boxes of sugar
between panels 3 and 4; and that the character intends to add sugar to the
pot.

%
%
%	 	Indeed, \citeA{saraceni2016relatedness} argues that comic structure
%	 	ought to demonstrate \emph{relatedness}, which spans from the
%	 	structural aspects of discourse to the cognitive aspects of discourse
%	 	(which turns out to be the structural aspect of stories, needed for
%	 	sense-making).
%
% \note{RCR}{Discuss the three elements of Saraceni's account, including the closure principle (third aspect of \emph{relatedness}, which depends on not what is overtly explicit in the narrative's surface code, but also on inference. Borrowed from visual perception and Gestalt principles.}

%\note{RCR}{I think a figure that illustrates the spectrum of
%structural-relatedness to cognitive-relatedness would be good here.}
%
In our work we sought to develop a small scale computational model, and thus
focused primarily on modeling discourse structure which lies on the textual
side of the spectrum. However, our discourse model includes a minimal model of
story, which is needed in order to account for some elements of the cognitive
side of the spectrum: in particular, we assume chronological ordering
between panels and track which visual elements have appeared previously in the
panel sequence.  We developed two computational models of discourse
structure: one based on McCloud's~\cite{mcCloud1993understanding}'s account
of \emph{transition types}, and the other based on
Cohn's~\cite{cohn2013visual} theory of visual language.


% Human brains' ability to fill in gaps is also why comics are simpler than
% animation in this respect: animations are expected to provide continuous
% motion between frames, whereas two comic frames need only be plausibly
% connected by some narrative justification. And that's where transition types
% come in: when you exclude non sequiturs, they constrain the space of next
% panels to ones that "make sense."

