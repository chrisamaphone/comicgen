%================================================================
\section{Comics Theory}

The basis of McCloud's theory about making sense of panel sequences across
the gutters, later validated experimentally, is that
readers of comics optimize their
consumption of relevant information~\cite{pirolli2007information}, and work
to construct inferences~\cite{magliano2016filling} about story content in
these liminal spaces of discourse.
% (in between sentences in text, panels in
% comics, scenes in film). 
Inferences for story content are constructed when
they are needed for comprehension and enabled by what has
been narrated thus far~\cite{myers1987degree}. The dynamic
between story authors and audiences parallels the dynamics of people
engaged in cooperative conversation as outlined by the philosopher of
language Grice~\cite{grice1975logic}: the storyteller, as the active contributor
to the ongoing communicative context, is expected to make her contributions
to the discourse based on what is relevant to her narrative intent. 
% As Murray~\cite{murray2011why} states:
% %
% \begin{quote} 
% 	In a mature medium nothing happens, nothing is brought on stage (or screen 
% 	or comic book panel or described in prose) that does not in some way further 
% 	the action. Whatever the viewer is invited to direct attention to is
% 	something that further defines the role ([of a] character) or the function 
% 	(dramatic beat). 
% 	\end{quote}
% %
% These expectations give rise to narrative devices such as \emph{Chekhov's
% gun}, wherein narrative elements are introduced because they are relevant, 
% and they ultimately demonstrate their relevance at some point in the story.
% Narrative authors can at the same time flout this expectation of 
% cooperativity in service of a counterpart narrative device, the 
% \emph{red herring}, wherein a story element is introduced and which
% ultimately has no relevance to the unfolding story.
% 
% Thus, for narrative generation, we seek to represent the constraints and
% effects of narrative discourse, since (as the primary point of contact with
% the narrative artifact) narrative discourse carries with it expectations
% and conventions that ultimately affect how story consumers understand the
% narrative. 
% 
% While coherent story structure is important for
% comprehension~\cite{graesser2002how}, the audience recovers that structure
% only insofar as it is afforded by the discourse structure. 

% {\em Purely visual comics}, or sequences of visual imagery arranged in
% panels, present an excellent avenue along which to study discourse
% theories computationally. 
% In comics, the same principles apply: comprehensible comics lack visual
% clutter, and differences across the gutters (gaps between panels) are
% designed to be filled in by an audience's inference.  
% These principles, as
% well as notions of brevity, relatedness, and other principles of
% cooperative narration, manifest in terms of discrete particles that may be
% recognized and generated by computer programs. 
McCloud's introduced six {\em
panel transition types} for comics~\cite{mcCloud1993understanding}, 
enumerate the different roles that the reader may infer from a
well-written comic. These transition types are {\em moment-to-moment}, {\em
action-to-action}, {\em subject-to-subject}, {\em aspect-to-aspect}, {\em
scene-to-scene}, and non sequitur. While it is tempting to think we could
simply operationalize these transitions in a generator, as
Cohn~\cite{cohn2013visual} (Chapter 4) points out, so much of their meaning
relies on contextual, real-world-situated understanding that it lends
little help to computational authoring.

% However, other scholars have identified that comics are structurally similar to written
% text~\cite{saraceni2016relatedness}: they are both made up of individual
% elements (sentences in text, panels in comics), delimited by special-purpose
% symbols (full stops in text, panel borders in comics), which can be easily
% identified, and which can contain a variable amount of information. However,
% unlike text, comics also express information via visual elements and their spatial
% relationships to one another, which we might model in terms of their
% relative size, rotation, horizontal and vertical juxtaposition, and
% distance. While in general comics offer two dimensions of authorship
% affordances (textual and visual language), in this work we are concerned
% only with the pictorial dimension.
%
\begin{figure}
	\centering
	\includegraphics[width=0.75\columnwidth]{relatedness.png}
	\caption{
		The spectrum of \emph{relatedness} as discussed by
                Saraceni~\cite{saraceni2016relatedness}. Relatedness indicates how 
		comic panels are connected or associated in the minds of 
		readers, spanning from textual factors to cognitive factors. 
		Along that spectrum, there are three  distinguished 
		categories of relatedness: \emph{repetition}, 
		\emph{collocation}, and \emph{closure}, which have
		demonstrably different effects on the construction of
		narrative mental models.
		}
	\label{figure:relatedness}
\end{figure}
%

On the other hand, Saraceni~\cite{saraceni2016relatedness} validated
McCloud's hunch that readers create meaning from comics from perceived
relationships between panels.
Saraceni describes three notions of \emph{relatedness} between comic
elements, which are the building blocks from which readers may construct
meaning inferences.  Relatedness, a property of a comic that indicates how
its panels are connected or associated, depends on a comic's
\emph{cohesion} -- the lexico-grammatical features that tie panels together
-- and \emph{coherence} -- the audience's perception of how individual
panels contribute to her mental model of the unfolding events. Relatedness
emerges from a spectrum of \emph{textual}\footnote{\emph{Textual} here does
not mean the use of actual text, but rather is a shorthand for
\emph{surface code}~\cite{zwaan1998situation}.} factors to \emph{cognitive}
factors, illustrated in \fref{figure:relatedness}.
%
Saraceni distinguishes three categories of relatedness.  Closer to the
textual end of the spectrum is the \emph{repetition} of visual elements
across panels. Beyond repetition is \emph{collocation}, which refers to an
audience's expectation that related visual elements will appear given the
ones that have been perceived. Closer to the cognitive end of the spectrum
is the \emph{closure} over comic elements, which refers to the way our
minds complete narrative material given to us. Closure is terminologically
borrowed from the field of visual cognition, but is intended as the mental
process of inference that occurs as part of an audience's \emph{search for
meaning}~\cite{gerrig1994readers}.

%
% \begin{figure}[t]
%   \centering
% 	\includegraphics[width=9cm]{xkcd-to_taste.png}
% 	\caption{
% 		Strip \#1639 of {\em XKCD}, {\small\copyright} Randall Munroe. This comic
% 		depends on three aspects of relatedness as described by 
%                 Saraceni~\cite{saraceni2016relatedness}, and as illustrated in 
% 		\fref{figure:relatedness}.
% 	}
% \label{fig:xkcd}
% \end{figure}
The comic in Figure~\ref{fig:calvin} depends on these three 
aspects of relatedness: first, repetition of the sled, snowman, and other
figures maintains cohesion across panels. Second, the humor of the sled
carrying off the snowman depends on our (non-grammatical) domain knowledge
that the snowman is not a living character in the same sense as the other
figures. Finally, the comic depends on closure for the audience to ``fill
in the gaps'' to infer what must have happened during the ``WUMP!'' panel:
the sled maintained momentum to carry off the snowman, and the riders of
the sled landed on top of the girl.

% of the stove and pot is used to maintain
% cohesion across panels. Second, the punchline of the comic depends on
% collocation in the sense that we expect ``sugar'' to come in small
% measurements, based on non-grammatical domain knowledge.  Finally, the
% comic depends on closure in that we expect the audience to infer several
% things: that before the start of the comic, the character had been
% following a recipe; that the character went to retrieve the boxes of sugar
% between panels 3 and 4; and that the character intends to add sugar to the
% pot.
% =======
% XXX consider adding back in.
% \begin{figure}[bt]
% 	\centering
% 	\includegraphics[width=0.8\columnwidth]{xkcd-to_taste.png}
% 	\caption{
% 		Strip \#1639 of {\em XKCD}, {\small\copyright} Randall Munroe. This comic
% 		depends on three aspects of relatedness as described by 
%                 Saraceni~\cite{saraceni2016relatedness}, and as illustrated in 
% 		\fref{figure:relatedness}.
% 	}
% \label{fig:xkcd}
% \end{figure}
% In Figure~\ref{fig:xkcd}, we see a comic that depends on the three aforementioned 
% aspects of relatedness: first, repetition of the stove and pot is used to maintain
% cohesion across panels. Second, the punchline of the comic depends on
% collocation in the sense that we expect ``sugar'' to come in small
% measurements, based on non-grammatical domain knowledge.  Finally, the
% comic depends on closure in that we expect the audience to infer several
% things: that before the start of the comic, the character had been
% following a recipe; that the character went to retrieve the boxes of sugar
% between panels 3 and 4; and that the character intends to add sugar to the
% pot.
% >>>>>>> recardona/writing

%
%
%	 	Indeed, \citeA{saraceni2016relatedness} argues that comic structure
%	 	ought to demonstrate \emph{relatedness}, which spans from the
%	 	structural aspects of discourse to the cognitive aspects of discourse
%	 	(which turns out to be the structural aspect of stories, needed for
%	 	sense-making).
%
% \note{RCR}{Discuss the three elements of Saraceni's account, including the closure principle (third aspect of \emph{relatedness}, which depends on not what is overtly explicit in the narrative's surface code, but also on inference. Borrowed from visual perception and Gestalt principles.}

In our work we sought to develop a small scale computational model, and thus
focused primarily on modeling discourse structure which lies on the textual
side of the spectrum. However, our discourse model includes a minimal model of
story, which is needed in order to account for some elements of the cognitive
side of the spectrum: in particular, we assume chronological ordering
between panels and track which visual elements have appeared previously in the
panel sequence.  We developed two compatible models of discourse
structure: one based on McCloud's transition types and the other based on
Cohn's~\cite{cohn2013visual} theory of visual language.


% Human brains' ability to fill in gaps is also why comics are simpler than
% animation in this respect: animations are expected to provide continuous
% motion between frames, whereas two comic frames need only be plausibly
% connected by some narrative justification. And that's where transition types
% come in: when you exclude non sequiturs, they constrain the space of next
% panels to ones that "make sense."

