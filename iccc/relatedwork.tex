%================================================================
\section{Related Work}

As discussed in the Introduction, the pipeline model of narrative generation 
has been the dominant paradigm to narrative generation. In this section we
review some exemplars of that model, with special focus on systems that have
been covered in the computational creativity community.

\citeA{guerrero2014social} developed a nuanced computational model of social
norms to drive the interaction of characters in the simulation of the story
world. Their work defers the realization of text to
MEXICA~\cite{perez2001mexica}, a computational implementation of a
cognitively-oriented account of writing. However MEXICA itself is primarily a
story-level reasoner, since it leaves unspecified how the story structures that
it generates via computational engagement and reflection are realized into
narrative text.

\citeA{montfort2013slant} developed a blackboard architecture called Slant for
story generation that integrates several different sub-components systems to
generate a story. While the system's architecture is primarily dedicated to the
specification and refinement of rules to generate plot structure, Slant does
include a sub-component called Verso, which reasons over narrative discourse as
a way to further constrain the narrative plot. In particular, Verso detects
aspects of the verbs used during the generation of plot structure, and
determines the in-progress story's match to a specific genre.\footnote{Verso's
operationalization of genre differs from the literary sense of the term, but a
full discussion of this is beyond the scope of our work.} Once a specific genre
has been identified, Verso poses additional constraints to the plot generator
via the Slant blackboard. Slant is thus not strictly a pipeline model
architecture, but unfortunately the constraints identified during discourse
reasoning cannot themselves inform further discourse reasoning. In our approach,
we hope to identify discourse-driven narrative generation that informs or
constrains both the generation of the underlying plot structure, as well as the
further generation of narrative discourse.

Most relevant to the work we pursue here is the work by
\citeA{perezyperez2012illustrating}, who developed a visual illustrator to their
MEXICA system. They sought to verify the degree to which their 3-panel comic 
generator elicited in readers the same sense of story as a textual realization 
of the same MEXICA-generated plot. While this system still follows the pipeline
model of narrative generation, we see their work as complementary: they developed 
an experiment methodology through which it is possible to empirically assess if
their palette of designed visual elements denote story concepts as intended. 
Future work in discourse-driven comic generation will have to address this
point going forward, and \citeauthor{perezyperez2012illustrating} provide a step
toward understanding the gap between story concepts and the computational symbols
meant to encode them. A potential improvement to their system that the authors 
identify as most important was: ``to provide the Visual Narrator with mechanisms
that allow more freedom during the composition 
process''~\cite{perezyperez2012illustrating}. Our work here aimed to provide just 
that.
