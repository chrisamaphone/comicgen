%================================================================
\section{Related Work}





\begin{itemize}

	\item Talk about Understanding Comics~\cite{mcCloud1993understanding}
	\item Talk about Visual Language of Comics~\cite{cohn2013visual}
	\item Talk about the MEXICA System~\cite{perez2001mexica} and how we're different
	\item Talk about the departure from traditional narrative generation work
	\begin{itemize}
		\item Talk about the pipeline model of narrative generation (primarily 
		simulation focused)
		\item We're exploring an alternative account - 
		focus on the telling of the story, let story consumers ``fill in the gaps''
	\end{itemize}

\end{itemize}



Historically, the computational generation of narrative has followed what \citeA{ronfard2014story} term the \emph{pipeline model}: a narrative artefact is computationally generated by first simulating the story world as a collection of events, and then piping the story world information to a discourse generator, which generates a selective presentation of story world events in a particular medium. Current work in the computational creativity community has primarily pursued this pipeline model for narrative generation. Work by \note{RCR}{There are several papers to cite here, but the gist is: pipeline model is pervasive.}

As \citeauthor{ronfard2014story} argue, the pipeline model is neither \emph{necessary} nor \emph{sufficient} for the successful generation of narrative structure, but rather represents one paradigm of narrative generation.  \note{RCR}{Present one example of why it's not necessary, and one example of why it's not sufficient.}


Our work presents a departure from the pipeline model, opting instead for a \emph{discourse-first approach to narrative generation}. In this model, the story world is simulated inasmuch as is necessary to support the telling of story events in the discourse. The work we present here is a first step in this discourse-driven model, focused on understanding how the discourse of visual language narratives enforces constraints on the underlying story worlds they represent, and how these can further guide subsequent choices for discursive presentation.

Narrative authors design their narratives to affect audiences in specific ways (Bordwell), and authors care to structure discourse to achieve specific narratological effects.  This is achievable in the pipeline model, but reasoning about discourse first helps you catch things that are not realizable in a specific target medium (e.g. identity for murderers in thrillers.)



