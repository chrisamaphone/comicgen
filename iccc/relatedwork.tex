%================================================================
\section{Related Work}

Historically, the computational generation of narrative has followed what
\citeA{ronfard2014story} term the \emph{pipeline model}: a narrative artifact is
computationally generated by first simulating the story world as a collection of
events, and then piping the story world information to a discourse generator,
which generates a selective presentation of story world events in a particular
medium. Current work in the computational creativity community has primarily
pursued this pipeline model for narrative generation. Work by\ldots


\note{RCR}{There are several papers to cite here, but the gist is: pipeline
model is pervasive.}


\begin{itemize}

	\item For all systems cited, compare and contrast. 
	\begin{itemize} 
		\item Talk about MEXICA~\cite{perez2001mexica}. 
		\item Talk about Slant~\cite{} 
	\end{itemize}

\end{itemize}

As \citeauthor{ronfard2014story} argue, the pipeline model is neither
\emph{necessary} nor \emph{sufficient} for the successful generation of
narrative structure. This is because authors intentionally design their
narratives to affect audiences in specific
ways~\cite{chatman1980story,bordwell1989making}, which involves reasoning beyond
what is communicated (the underlying story world) but rather how it is
communicated. It is unnecessary to simulate an aspect of the narrative universe
that is never communicated to the audience, if it does not inform the ultimate
delivery of the narrative artifact. Similarly, it is insufficient to reason
about the story and discourse constituents independent of each other, because
the characteristics of a discourse realization shape the stories that can be
told in that medium~\cite{herman2004toward}. As mentioned earlier, constraining
the generation process to the pipeline model unnecessarily restricts how
creative the generator can ultimately be, since story world commitments are not
revisited when generating discourse.

Our work presents a departure from the pipeline model, opting instead for a
\emph{discourse-first approach to narrative generation}. In this model, the
story world is simulated inasmuch as is necessary to support the telling of
story events in the discourse. The work we present here is a first step in this
discourse-driven model, focused on understanding how the discourse of visual
language narratives enforces constraints on the underlying story worlds they
represent, and how these can further guide subsequent choices for discursive
presentation.



