%================================================================
\section{Introduction}






\begin{itemize}

	\item What are we trying to do?
	\item What is our approach?
	\item Talk about how creative the discipline is
	\item Why does discourse-driven help creativity in narrative generation?
		\begin{itemize}
			\item Creative potential is bigger in discourse-driven approach, principle 
			of least-commitment in the fabula - don't need to simulate a portion of 
			the narrative universe which is never revealed which may impose limitations
			on future narrative generation. \note{RCR}{Example!}
			
			\item Focusing on the telling may leave aspects of fabula unspecified, which 
			may broaden the interpretation of the story in the minds of story consumers. 
			\note{RCR}{Example!}
			\item Talk about the pipeline model of narrative generation (primarily
				simulation focused)
			\item We're exploring an alternative account - focus on the telling of the 
				story, let story consumers ``fill in the gaps''
		
			\begin{itemize}
				\item Gricean Maxims
				\item Closure principle Saraceni - Third aspect of ``relatedness'', 
				depends on not what is overtly explicit in the narrative's surface 
				code, but also on inference.  ``Closure'' comes from Gestalt principles.
				\item Gestalt principle 
			\end{itemize}	
		\end{itemize}
	\item Why are comics a great domain for computational creativity?
	
	\item Talk about Understanding Comics~\cite{mcCloud1993understanding}
	\item Talk about Visual Language of Comics~\cite{cohn2013visual}
\end{itemize}

We pursued a discourse-driven 

