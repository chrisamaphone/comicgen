%================================================================
\section{Introduction}

The computational generation of stories is an enterprise that can help us
precisely understand one of the most creative aspects of human
intelligence~\cite{boyd2009origin}: the ability to both generate stories about
possible and impossible worlds, and the ability to weave stories around our
daily life~\cite{herman2013storytelling}. In this paper, we begin to explore a
relatively unexplored sub-domain of computational narrative generation:~the
computational generation of comics. This targeted exploration is in the form of
a \emph{small-scale computational system}~\cite{montfort2012small} designed to
model what we believe are two key aspects of computationally creative narrative
generation:
%
\begin{inparaenum}[]
	\item a discourse-first orientation, and
	\item the importance of both hierarchical and sequential reasoning.
\end{inparaenum}


The computational generation of comics presents a novel, yet approachable
challenge. Comics share structural similarity to
text~\cite{saraceni2016relatedness}: they are both made up of individual
elements (sentences in text, frames in comics), delimited by special-purpose
symbols (full stops in text, frame borders in comics), which can be easily
identified, and which can contain a variable amount of information. However,
unlike text, comics afford an additional \emph{pictorial} dimension through
which to express information. To \citeA{cohn2013visual}, comics are one instance
of the more broader \emph{theory of visual language}, which communicates via a
palette of visual elements and their spatial relationships to one another, e.g.
their relative size, rotation, horizontal and vertical juxtaposition, and
distance. 



% Human brains' ability to fill in gaps is also why comics are simpler than animation in this respect: animations are expected to provide continuous motion between frames, whereas two comic frames need only be plausibly connected by some narrative justification. And that's where transition types come in: when you exclude non sequiturs, they constrain the space of next panels to ones that "make sense."













\begin{itemize}
	\item Why are comics a great domain for computational creativity?
	\begin{itemize}
		\item Talk about how creative the discipline is
		\item Motivated by the exploration of computational creativity to novel domains
	\end{itemize}
	\item What are we trying to do?
	\item What is our approach?
	\item Why does discourse-driven help creativity in narrative generation?
		\begin{itemize}
			\item Creative potential is bigger in discourse-driven approach, principle 
			of least-commitment in the fabula - don't need to simulate a portion of 
			the narrative universe which is never revealed which may impose limitations
			on future narrative generation. \note{RCR}{Example!}
			
			\item Focusing on the telling may leave aspects of fabula unspecified, which 
			may broaden the interpretation of the story in the minds of story consumers. 
			\note{RCR}{Example!}
			\item Talk about the pipeline model of narrative generation (primarily
				simulation focused)
			\item We're exploring an alternative account - focus on the telling of the 
				story, let story consumers ``fill in the gaps''
		
			\begin{itemize}
				\item Gricean Maxims
				\item Closure principle Saraceni - Third aspect of ``relatedness'', 
				depends on not what is overtly explicit in the narrative's surface 
				code, but also on inference.  ``Closure'' comes from Gestalt principles.
				\item Gestalt principle 
			\end{itemize}	
		\end{itemize}

	
	\item Talk about Understanding Comics~\cite{mcCloud1993understanding}
	\item Talk about Visual Language of Comics~\cite{cohn2013visual}
\end{itemize}

We pursued a discourse-driven 

