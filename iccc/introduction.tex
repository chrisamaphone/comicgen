%================================================================
\section{Introduction}

The computational generation of stories (hereafter {\em narrative
generation}) is an enterprise that can help us understand one of the most
creative aspects of human intelligence~\cite{boyd2009origin}: reasoning
about possible and impossible worlds, and weaving narratives around our
daily lives~\cite{herman2013storytelling}.

Historically, narrative generation has followed what
\citeA{ronfard2014story} term the \emph{pipeline model}: a narrative artifact is
computationally generated by first simulating the story world as a collection of
events, and then piping the story world information to a discourse generator, 
which generates a selective presentation of story world events in a particular 
medium. While we defer our discussion of related  work until a later section, 
a great deal of existing work in the computational creativity community has 
primarily pursued this pipeline model for narrative 
generation~\cite{gervas2009computational}. 

As \citeauthor{ronfard2014story} argue, the pipeline model is neither
\emph{necessary} nor \emph{sufficient} for narrative generation. 
% \note{CRM}{Should we explain what ``successful'' means?}
% RCR: Good point, removed vague term.
Authors intentionally design their narratives to affect audiences in
specific ways~\cite{chatman1980story,bordwell1989making}, which involves
reasoning beyond what is communicated (the underlying story world) but
rather how it is communicated. It is unnecessary to simulate an aspect of
the narrative universe that is never communicated to the audience, if it
does not inform the ultimate delivery of the narrative artifact. It is also
insufficient to reason about the story and discourse constituents
independent of each other, because the characteristics of a discourse
realization shape the stories that can be told in that
medium~\cite{herman2004toward}. Constraining the generation process to the
pipeline model unnecessarily restricts how creative the generator can
ultimately be, since story world commitments are not revisited when
generating discourse. Further, as will be detailed later, narrative
authorship depends on audiences being able to fill in the gaps left
open in the consumption of a
story~\cite{saraceni2016relatedness,magliano2016filling}.

Our work presents a departure from the pipeline model, opting instead for a
\emph{discourse-driven approach to narrative generation}. In this model, the
story world is simulated inasmuch as is necessary to support the telling of
story events in the discourse. We present a \emph{small-scale computational
system}~\cite{montfort2012small} for discourse-driven comic generation, a
relatively unexplored domain of computational narratology~\cite{mani2012computational}.

In the remainder of this paper, we discuss theoretical aspects of comic
writing, our computational implementation of a comic generation system, and
our experience with the refinement of our model. Our primary takeaway is
that both global and local reasoning are important aspects of
narrative generation: local reasoning is important for maintaining
narrative coherence, and global reasoning is important for
maintaining satisfying narrative structure. Both are thus important parts
of creating what we feel are more comprehensible comics, and we present an
outline of future work designed to explore the human interpretation of our
generated artifacts.



