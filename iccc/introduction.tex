%================================================================
\section{Introduction}

The computational generation of stories is an enterprise that can help us
precisely understand one of the most creative aspects of human
intelligence~\cite{boyd2009origin}: the ability to both generate stories about
possible and impossible worlds, and the ability to weave stories around our
daily life~\cite{herman2013storytelling}.

Historically, the computational generation of narrative has followed what
\citeA{ronfard2014story} term the \emph{pipeline model}: a narrative artifact is
computationally generated by first simulating the story world as a collection of
events, and then piping the story world information to a discourse generator,
which generates a selective presentation of story world events in a particular
medium. Existing work in the computational creativity community has primarily
pursued this pipeline model for narrative
generation~\cite{gervas2009computational}. Work by\ldots


\note{RCR}{There are several papers to cite here, but the gist is: pipeline
model is pervasive.}

\citeA{jhala2010cinematic} developed\ldots


\citeA{guerrero2014social} developed a nuanced computational model of social
norms to drive the interaction of characters in the simulation of the story
world. Their work defers the realization of text to
MEXICA~\cite{perez2001mexica}, a computational implementation of a
cognitively-oriented account of writing. However MEXICA itself is primarily a
story-level reasoner, since it leaves unspecified how the story structures that
it generates via computational engagement and reflection are realized into
narrative text.

\citeA{montfort2013slant} developed a blackboard architecture called Slant for
story generation that integrates several different sub-components systems to
generate a story. While the system's architecture is primarily dedicated to the
specification and refinement of rules to generate plot structure, Slant does
include a sub-component called Verso, which reasons over narrative discourse as
a way to further constrain the narrative plot. In particular, Verso detects
aspects of the verbs used during the generation of plot structure, and
determines the in-progress story's match to a specific genre.\footnote{Verso's
operationalization of genre differs from the literary sense of the term, but a
full discussion of this is beyond the scope of our work.} Once a specific genre
has been identified, Verso poses additional constraints to the plot generator
via the Slant blackboard. Slant is thus not strictly a pipeline model
architecture, but unfortunately the constraints identified during discourse
reasoning cannot themselves inform further discourse reasoning. In our approach,
we hope to identify discourse-driven narrative generation that informs or
constrains both the generation of the underlying plot structure, as well as the
further generation of narrative discourse.

Most relevant to the work we pursue here is the work by
\citeA{perezyperez2012illustrating}, who developed\ldots


As \citeauthor{ronfard2014story} argue, the pipeline model is neither
\emph{necessary} nor \emph{sufficient} for the successful generation of
narrative structure. This is because authors intentionally design their
narratives to affect audiences in specific
ways~\cite{chatman1980story,bordwell1989making}, which involves reasoning beyond
what is communicated (the underlying story world) but rather how it is
communicated. It is unnecessary to simulate an aspect of the narrative universe
that is never communicated to the audience, if it does not inform the ultimate
delivery of the narrative artifact. Similarly, it is insufficient to reason
about the story and discourse constituents independent of each other, because
the characteristics of a discourse realization shape the stories that can be
told in that medium~\cite{herman2004toward}. Constraining the generation process
to the pipeline model unnecessarily restricts how creative the generator can
ultimately be, since story world commitments are not revisited when generating
discourse.

Our work presents a departure from the pipeline model, opting instead for a
\emph{discourse-driven approach to narrative generation}. In this model, the
story world is simulated inasmuch as is necessary to support the telling of
story events in the discourse. We present a \emph{small-scale computational
system}~\cite{montfort2012small} for discourse-driven comic generation, designed
to explore discourse-driven narrative generation in a relatively unexplored
sub-domain of computational narrative generation. In the remainder of this
paper, we discuss theoretical aspects of comic writing, our computational
implementation of a comic generation system, and our experience with the
refinement of our model. We present our primary takeaway that both hierarchical
and sequential reasoning are important parts of creating what we feel are more
comprehensible comics, and present an outline of future work designed to explore
the human interpretation of our generated artifacts.


for two reasons: \begin{itemize} \item Explore discourse-driven narrative
generation in a relatively unexplored sub-domain of computational narrative
generation, and \item Ill \end{itemize}

 demonstrate the importance of both hierarchical and sequential reasoning in the
 generation of visual narrative.




 which represents a relatively unexplored sub-domain of the more broader
 computational narrative generation.


as a proof-of-concept system in


for the purpose of


designed to model what we believe to be a key aspect of computationally creative
narrative generation





In this paper, we begin to explore a relatively unexplored sub-domain of
computational narrative generation:~the computational generation of comics. This
targeted exploration is in the form of a \emph{small-scale computational
system}~\cite{montfort2012small} designed to model what we believe are two key
aspects of computationally creative narrative generation:
%
\begin{inparaenum}[] \item a discourse-first orientation, and \item the
importance of both hierarchical and sequential reasoning. \end{inparaenum}










The work we present here is a first step in this discourse-driven model, focused
on understanding how the discourse of visual language narratives enforces
constraints on the underlying story worlds they represent, and how these can
further guide subsequent choices for discursive presentation.





























The computational generation of comics presents a novel, yet approachable
challenge. Comics share structural similarity to
text~\cite{saraceni2016relatedness}: they are both made up of individual
elements (sentences in text, frames in comics), delimited by special-purpose
symbols (full stops in text, frame borders in comics), which can be easily
identified, and which can contain a variable amount of information. However,
unlike text, comics afford an additional \emph{pictorial} dimension through
which to express information. To \citeA{cohn2013visual}, comics are one instance
of the more broader \emph{theory of visual language}, which communicates via a
palette of visual elements and their spatial relationships to one another, e.g.
their relative size, rotation, horizontal and vertical juxtaposition, and
distance.



% Human brains' ability to fill in gaps is also why comics are simpler than
% animation in this respect: animations are expected to provide continuous
% motion between frames, whereas two comic frames need only be plausibly
% connected by some narrative justification. And that's where transition types
% come in: when you exclude non sequiturs, they constrain the space of next
% panels to ones that "make sense."
% 
% 
% 
% 
% 
% 
% 
% 
% 
% 
% 
% 
% 
\begin{itemize} \item Why are comics a great domain for computational
creativity? \begin{itemize} \item Talk about how creative the discipline is
\item Motivated by the exploration of computational creativity to novel domains
\end{itemize} \item What are we trying to do? \item What is our approach? \item
Why does discourse-driven help creativity in narrative generation?
\begin{itemize} \item Creative potential is bigger in discourse-driven approach,
principle of least-commitment in the fabula - don't need to simulate a portion
of the narrative universe which is never revealed which may impose limitations
on future narrative generation. \note{RCR}{Example!}

			\item Focusing on the telling may leave aspects of fabula
			unspecified, which may broaden the interpretation of the story in
			the minds of story consumers. \note{RCR}{Example!} \item Talk about
			the pipeline model of narrative generation (primarily simulation
			focused) \item We're exploring an alternative account - focus on the
			telling of the story, let story consumers ``fill in the gaps''

			\begin{itemize} \item Gricean Maxims \item Closure principle
			Saraceni - Third aspect of ``relatedness'', depends on not what is
			overtly explicit in the narrative's surface code, but also on
			inference.  ``Closure'' comes from Gestalt principles. \item Gestalt
			principle \end{itemize} \end{itemize}


	\item Talk about Understanding Comics~\cite{mcCloud1993understanding} \item
	Talk about Visual Language of Comics~\cite{cohn2013visual} \end{itemize}

We pursued a discourse-driven

