Comics use sequences of still, two-dimensional visual stimuli to tell
stories, and they are deeply entwined with humanity's creative history.
While there is prior work on computationally generating discourse
(conveyance of narrative) for textual stories, few attempts have been made
to target comics, which carry distinct challenges and affordances, as a
discourse language.  Standard pipeline-based approaches to narrative
generation, wherein a discourse is configured only after a fabula has been
fixed, do not map well onto the affordances of purely visual panel
sequences.

% XXX can we back up this claim more strongly?

% Prior efforts toward comic-authoring computer programs mainly
% conform to fixed panel contents which cannot be flexibly rearranged
% while retaining narrative coherence.

% solution we propose
We propose a {\em discourse-first} scheme for generating comics, including novel configurations
of panel contents, by combining a
bottom-up panel sequencer inspired by McCloud's taxonomy of panel
transitions and a top-down grammar of comic arcs devised by Cohn.
% results
Our approach yields a flexible algorithm for comic generation whose output
can be rendered with a number of visual palettes, for which we have built
one proof-of-concept example and observed a wide range of variability in
output. Our efforts represent a first step toward robust comic generation
and provide a platform for analyzing and evaluating the validity of visual
discourse theories.


