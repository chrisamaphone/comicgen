% State of the art
Storytelling through images is a fundamental aspect of creativity, to which
computational narrative generation lends understanding by operationalizing
theories about how stories are constructed.  Current approaches to story
generation employ a pipeline model where discourse (delivery of narrative)
% problem
is built atop an underlying fabula structure. But a fabula-first approach
to story generation demands extensive, deep world modeling, much of which
effort will be ignored at discourse-generation time. 

% problem 2
Meanwhile, {\em
purely-visual comics} are a storytelling medium that uses still,
two-dimensional visual stimuli to tell stories, dramatically constraining
the language of discourse while opening up other narrative affordances.
Very few attempts have been made to computationally generate comics, mainly
conforming to fixed panel contents which cannot be flexibly rearranged
while retaining narrative coherence.
% standard fabula-based approaches are a poor match for their expressive
% range. (XXX can we back up this claim?)

% solution we propose
We propose a scheme for generating comics, including panel internals, based
on a discourse-first approach: each panel generated at the end of a
sequence is constrained by the history of prior panels (at the local level)
and, optionally, by a discourse grammar (at the global level).
% results
As a result, we have built a flexible framework for comic generation that
can be extended with a number of visual palettes. Our algorithm generates a
diverse range of comics that provide a vector for analyzing and evaluating
the validity of the visual discourse theories that drive them.

