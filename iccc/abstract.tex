% Narrative generation is important
Narrative generation enables a well-documented range of opportunities for
understanding the creative act of storytelling and employing it in digital
applications.
% but standard pipeline model has limits
Prior approaches have mostly converged on a ``pipeline'' model, wherein
fabula (event structure) is gengerated first, and discourse (means of
narrative delivery to an audience) is considered after the fact, mapping
individual fabula events to discourse elements. However, Ronfard and Szilas have
argued that the pipeline model severely limits
% our approach 
narrative possibilities. We investigate a new approach to
narrative generation based purely on discourse theories. In particular, 
% Comics offer good discourse theory
because {\em comics} offer a rich body of discourse theory, we built a
comic generator to study narrative generation.  In contrast to prior comic
generation work, we generate panels down to the level of visual element
positions within panels, rather than rearranging pre-existing panels.  Our
approach is based on leading discourse theories for comics given by McCloud
(panel transitions) and Cohn (narrative grammar for comics).
% results
Our contributions are a proof-of-concept generator with a wide range of
abstract comic output, a computational realization of McCloud's and Cohn's
comics theories, and a modular algorithm that provides a platform for
analyzing and evaluating the validity of visual discourse theories.


