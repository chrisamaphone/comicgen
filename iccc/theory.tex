%================================================================
\section{On Generating Comics}

The computational generation of comics presents a novel challenge. Comics 
share structural similarity to written text~\cite{saraceni2016relatedness}: 
they are both made up of individual elements (sentences in text, frames in 
comics), delimited by special-purpose symbols (full stops in text, frame 
borders in comics), which can be easily identified, and which can contain a 
variable amount of information. However, unlike text, comics afford an additional \emph{pictorial} dimension through which to express information. To 
\citeA{cohn2013visual}, comics are one instance of the more broader 
\emph{theory of visual language}, which itself is used to communicate via
a palette of visual elements and their spatial relationships to one another,
e.g. their relative size, rotation, horizontal and vertical juxtaposition, 
and distance. While in general comics offer two authorship affordances 
(textual and visual language), in this work we are concerned only with 
the pictorial dimension.

One interesting and paradoxical aspect of purely-visual comics is how
constrained they for conveying narrative structure. The constraint is 
two-fold. Firstly, there is a practical constraint in that there are 
only so many visual elements that can be placed in a particular frame
before the ensemble becomes too littered to understand. Secondly, there
is an attentional constraint in that (even if we were to mediate
the first concern) a potential story consumer could be potentially
disengaged at the prospect of having to parse an overwhelming amount of
visual detail. There thus exists a tacit expectation on behalf of story
consumers that authors will make their contributions to the narrative arc
as brief and relevant as they need to be, in a manner parallel to the
expected conduct of people engaged in cooperative conversation as outlined 
by \citeA{grice1975logic}.

Thus, for the enterprise of computational narratology, it seems prudent to
understand and computationally model the constraints of discourse, since 
it is the primary point of contact with the narrative artifact. While we 
acknowledge that story structure is important for 
comprehension~\cite{graesser2002how}, we feel that the story exists primarily
in the head of the narrative consumer, and that story structure is recovered 
by the consumer insofar it is afforded by the discourse structure. In particular
for the case of comics, \citeA{saraceni2016relatedness} argues that a comic's
discursive structure ought to afford readers a sense of \emph{relatedness}, 
between comic elements.

%	 	Indeed, \citeA{saraceni2016relatedness} argues that comic structure 
%		ought to demonstrate \emph{relatedness}, which spans from the structural
%		aspects of discourse to the cognitive aspects of discourse (which turns
%		out to be the structural aspect of stories, needed for sense-making).


\note{RCR}{I think a figure that illustrates the spectrum of structural-relatedness to cognitive-relatedness would be good here.}

In our work we sought to develop a small scale computational model, and thus focused primarily on modeling discourse structure which lies on the structural side of the spectrum. However, our discourse model includes a minimal model of story, which is needed in order to account for some elements of the cognitive side of the spectrum.  We developed two computational models of discourse structure: one based on\citeauthor{mcCloud1993understanding}'s~(\citeyear{mcCloud1993understanding}) account in his seminal book \emph{Understanding Comics: The Invisible Art}, and the other based on \citeauthor{cohn2013visual}'s~(\citeyear{cohn2013visual}) account based on his \emph{theory of visual language}.


% Human brains' ability to fill in gaps is also why comics are simpler than
% animation in this respect: animations are expected to provide continuous
% motion between frames, whereas two comic frames need only be plausibly
% connected by some narrative justification. And that's where transition types
% come in: when you exclude non sequiturs, they constrain the space of next
% panels to ones that "make sense."
% 
% 
% 
% 
% 
% 
% 
% 
% 
% 
% 
% 
% 
\begin{itemize} \item Why are comics a great domain for computational
creativity? \begin{itemize} \item Talk about how creative the discipline is
\item Motivated by the exploration of computational creativity to novel domains
\end{itemize} \item What are we trying to do? \item What is our approach? \item
Why does discourse-driven help creativity in narrative generation?
\begin{itemize} \item Creative potential is bigger in discourse-driven approach,
principle of least-commitment in the fabula - don't need to simulate a portion
of the narrative universe which is never revealed which may impose limitations
on future narrative generation. \note{RCR}{Example!}

			\item Focusing on the telling may leave aspects of fabula
			unspecified, which may broaden the interpretation of the story in
			the minds of story consumers. \note{RCR}{Example!} \item Talk about
			the pipeline model of narrative generation (primarily simulation
			focused) \item We're exploring an alternative account - focus on the
			telling of the story, let story consumers ``fill in the gaps''

			\begin{itemize} \item Gricean Maxims \item Closure principle
			Saraceni - Third aspect of ``relatedness'', depends on not what is
			overtly explicit in the narrative's surface code, but also on
			inference.  ``Closure'' comes from Gestalt principles. \item Gestalt
			principle \end{itemize} \end{itemize}


	\item Talk about Understanding Comics~\cite{mcCloud1993understanding} \item
	Talk about Visual Language of Comics~\cite{cohn2013visual} \end{itemize}

We pursued a discourse-driven

