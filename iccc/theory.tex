%================================================================
\section{On Generating Comics}

One interesting and paradoxical aspect of narrative discourse is how constrained
it is for conveying narrative structure. The constraint is twofold. For
starters, there is a practical constraint in that there are only so many
discourse elements that can be delivered at a time before the ensemble becomes
too littered to understand. In visual media, there are only so many visual
elements that can be portrayed at any one instance (either in a frame, or on a
screen) before the discourse becomes incomprehensible. Textual media does not
necessarily suffer from this constraint, but it suffers from the other: the
attentional limits of the story consumer. A reader or viewer could be
potentially overtaxed at having to parse an overwhelming amount of narrative
detail.

There thus exists a tacit expectation on behalf of story consumers that authors
will make their contributions to the narrative arc as brief and relevant as they
need to be. As \citeA{murray2011why} states:
%
\begin{quote} In a mature medium nothing happens, nothing is brought on stage
(or screen or comic book panel or described in prose) that does not in some way
further the action. Whatever the viewer is invited to direct attention to is
something that further defines the role (character) or the function (dramatic
beat). \end{quote}
%
These expectations give rise to narrative devices such as \emph{Chekhov's gun},
and a related counterpart, the \emph{red herring}.

By the same token, authors construct narratives in a way that facilitates their
comprehension by the audience. People learn to optimize their consumption of
relevant information~\cite{pirolli2007information}, and work to construct
inferences~\cite{magliano2016filling} about story content in the liminal spaces
of discourse (in between sentences in text, panels in comics, scenes in film);
inferences for story content are constructed when they are \emph{needed} for
comprehension, and \emph{enabled} by what has been narrated thus
far~\cite{myers1987degree}. All told, the dynamic between story authors and
consumers parallels the expected conduct of people engaged in cooperative
conversation as outlined by the philosopher of language \citeA{grice1975logic}.

Thus, for the enterprise of computational narratology, it seems prudent to
understand and computationally model the constraints and effects of narrative
discourse, since (as the primary point of contact with the narrative artifact)
narrative discourse carries with it expectations and conventions that ultimate
effect how story consumers understand the narrative. While we acknowledge that
coherent story structure is important for comprehension~\cite{graesser2002how},
we feel that the story exists primarily in the head of the narrative consumer,
and that story structure is recovered by the consumer insofar it is afforded by
the discourse structure. Because theories of discourse structure feature
prominently in the study of purely-visual comics, the computational generation
of comics presents an excellent avenue through which to begin to study the
computational role of narrative discourse in the generation of narrative.

Comics share structural similarity to written
text~\cite{saraceni2016relatedness}: they are both made up of individual
elements (sentences in text, panels in comics), delimited by special-purpose
symbols (full stops in text, panel borders in comics), which can be easily
identified, and which can contain a variable amount of information. However,
unlike text, comics afford an additional \emph{pictorial} dimension through
which to express information via a palette of visual elements and their spatial
relationships to one another, e.g. their relative size, rotation, horizontal and
vertical juxtaposition, and distance. While in general comics offer two
dimensions of authorship affordances (textual and visual language), in this work
we are concerned only with the pictorial dimension.

\citeauthor{saraceni2016relatedness} argues that a comic's discursive structure
ought to afford readers a sense of \emph{relatedness} between comic elements.
Relatedness, a property of a comic that indicates how constituent panels are
connected or associated, depend on a comic's \emph{cohesion} -- the
lexico-grammatical features that tie panels together -- and \emph{coherence} --
the reader's perception of how individual panels contribute to her mental model
of the unfolding events. Relatedness emerges from a spectrum of \emph{textual}
factors to \emph{cognitive} factors, illustrated in \fref{figure:relatedness}.

%Closer to the textual end of that spectrum is the \emph{repetition} of visual
%elements across panels. 

%
\begin{figure}
	\includegraphics[width=\columnwidth]{relatedness.png}
	\caption{
		The spectrum of \emph{relatedness} as discussed by
		\citeA{saraceni2016relatedness}. Relatedness indicates how 
		comic panels are connected or associated in the minds of 
		readers, spanning from textual factors to cognitive factors. 
		Along that spectrum, there are three  distinguished 
		categories of relatedness: \emph{repetition}, 
		\emph{collocation}, and \emph{closure}, which have
		demonstrably different effects on the construction of
		narrative mental models.
		}
	\label{figure:relatedness}
\end{figure}
%



%	 	Indeed, \citeA{saraceni2016relatedness} argues that comic structure
%	 	ought to demonstrate \emph{relatedness}, which spans from the
%	 	structural aspects of discourse to the cognitive aspects of discourse
%	 	(which turns out to be the structural aspect of stories, needed for
%	 	sense-making).
%
\note{RCR}{Discuss the three elements of Saraceni's account, including the
closure principle (third aspect of \emph{relatedness}, which depends on not what
is overtly explicit in the narrative's surface code, but also on inference. 
Borrowed from visual perception and Gestalt principles.}


%\note{RCR}{I think a figure that illustrates the spectrum of
%structural-relatedness to cognitive-relatedness would be good here.}
%
In our work we sought to develop a small scale computational model, and thus
focused primarily on modeling discourse structure which lies on the structural
side of the spectrum. However, our discourse model includes a minimal model of
story, which is needed in order to account for some elements of the cognitive
side of the spectrum.  We developed two computational models of discourse
structure: one based
on~\citeauthor{mcCloud1993understanding}'s~(\citeyear{mcCloud1993understanding})
account of \emph{transition types}, and the other based on
\citeauthor{cohn2013visual}'s~(\citeyear{cohn2013visual}) \emph{theory of visual
language}.


% Human brains' ability to fill in gaps is also why comics are simpler than
% animation in this respect: animations are expected to provide continuous
% motion between frames, whereas two comic frames need only be plausibly
% connected by some narrative justification. And that's where transition types
% come in: when you exclude non sequiturs, they constrain the space of next
% panels to ones that "make sense."

