%================================================================
\section{On Generating Comics}

The computational generation of comics presents a novel challenge. Comics 
share structural similarity to written text~\cite{saraceni2016relatedness}: 
they are both made up of individual elements (sentences in text, panels in 
comics), delimited by special-purpose symbols (full stops in text, panel 
borders in comics), which can be easily identified, and which can contain a 
variable amount of information. However, unlike text, comics afford an additional \emph{pictorial} dimension through which to express information. To 
\citeA{cohn2013visual}, comics are one instance of the more broader 
\emph{theory of visual language}, which itself is used to communicate via
a palette of visual elements and their spatial relationships to one another,
e.g. their relative size, rotation, horizontal and vertical juxtaposition, 
and distance. While in general comics offer two dimensions of authorship 
affordances (textual and visual language), in this work we are concerned 
only with the pictorial dimension.

One interesting and paradoxical aspect of purely-visual comics is how
constrained they for conveying narrative structure. The constraint is 
two-fold. Firstly, there is a practical constraint in that there are 
only so many visual elements that can be placed in a particular frame
before the ensemble becomes too littered to understand. Secondly, there
is an attentional constraint in that (even if we were to mediate
the first concern) a potential story consumer could be potentially
disengaged at the prospect of having to parse an overwhelming amount of
visual detail. There thus exists a tacit expectation on behalf of story
consumers that authors will make their contributions to the narrative arc
as brief and relevant as they need to be. As \citeA{murray2011why} states:
%
\begin{quote}
	In a mature medium nothing happens, nothing is brought on stage (or 
	screen or comic book panel or described in prose) that does not in
	some way further the action. Whatever the viewer is invited to direct
	attention to is something that further defines the role (character) 
	or the function (dramatic beat).
\end{quote}
%
These expectations give rise to narrative devices such as \emph{Chekhov's gun},\footnote{\url{http://tvtropes.org/pmwiki/pmwiki.php/Main/ChekhovsGun}} and a
related counterpart, the \emph{red herring}.\footnote{\url{http://tvtropes.org/pmwiki/pmwiki.php/Main/RedHerring}}

By the same token, authors construct narratives in a way that
facilitates their comprehension by the audience. People learn to optimize 
their consumption of relevant information~\cite{pirolli2007information}, 
and work to construct inferences~\cite{magliano2016filling} about story 
content in the liminal spaces of discourse (in between sentences in text, 
and in between panels in comics); inferences of story content are 
constructed when they are \emph{needed} for comprehension, and 
\emph{enabled} by what has been told thus far~\cite{myers1987degree}.

All told, the dynamic between story authors and consumers parallels the
expected conduct of people engaged in cooperative conversation as outlined
by the philosopher of language \citeA{grice1975logic}.

\note{RCR}{An example comic here would be fantastic.}

Thus, for the enterprise of computational narratology, it seems prudent to
understand and computationally model the constraints and effects of narrative
discourse, since (as the primary point of contact with the narrative artifact) 
narrative discourse carries with it expectations and conventions that ultimate
effect how story consumers understand the narrative. While we 
acknowledge that coherent story structure is important for 
comprehension~\cite{graesser2002how}, we feel that the story exists primarily
in the head of the narrative consumer, and that story structure is recovered 
by the consumer insofar it is afforded by the discourse structure. In particular
for the case of comics, \citeA{saraceni2016relatedness} argues that a comic's
discursive structure ought to afford readers a sense of \emph{relatedness}, 
between comic elements.

%	 	Indeed, \citeA{saraceni2016relatedness} argues that comic structure 
%		ought to demonstrate \emph{relatedness}, which spans from the structural
%		aspects of discourse to the cognitive aspects of discourse (which turns
%		out to be the structural aspect of stories, needed for sense-making).

\note{RCR}{Discuss the three elements of Saraceni's account, including the closure principle (third aspect of \emph{relatedness}, which depends on not what is overtly explicit in the narrative's surface code, but also on inference.  Borrowed from visual perception and Gestalt principles.}


\note{RCR}{I think a figure that illustrates the spectrum of structural-relatedness to cognitive-relatedness would be good here.}

In our work we sought to develop a small scale computational model, and thus focused primarily on modeling discourse structure which lies on the structural side of the spectrum. However, our discourse model includes a minimal model of story, which is needed in order to account for some elements of the cognitive side of the spectrum.  We developed two computational models of discourse structure: one based on~\citeauthor{mcCloud1993understanding}'s~(\citeyear{mcCloud1993understanding}) account of \emph{transition types}, and the other based on \citeauthor{cohn2013visual}'s~(\citeyear{cohn2013visual}) \emph{theory of visual language}.


% Human brains' ability to fill in gaps is also why comics are simpler than
% animation in this respect: animations are expected to provide continuous
% motion between frames, whereas two comic frames need only be plausibly
% connected by some narrative justification. And that's where transition types
% come in: when you exclude non sequiturs, they constrain the space of next
% panels to ones that "make sense."

