\section{System Description}

Our approach to generating visual narratives begins as a linear
process that selects next comic panels based on the contents of previous
panels, choosing randomly among indistinguishably-valid choices.
The concepts we represent formally are {\em transitions}, {\em frames}, and
{\em visual elements}, which we define below.

% XXX how to make this a heading that looks different from a section
% heading?
% \subsection{Visual elements, frames, and transitions}

A {\bf visual element (VE)} is a unique identifier from an infinite set,
each of which is possible to map to a distinct visual representation.
We do not explicitly tag visual elements as specifically characters, props,
or scenery, making the representation agnostic to which of these narrative
interpretations will apply. In the visual rendering of our comics, we
represent VEs as random combinations of shape, color, and size, supplying
additional inputs to the human cognitive processes that may interpret these
elements' narrative role.

A {\bf frame} is a panel template; at the abstract generation level, it
includes an identifier or set of tags and a minimum number of required
visual elements. The reason a frame specifies a {\em minimum} number of VEs
is to allow for augmentation of the frame with pre-existing elements: for
example, the {\em monologue} frame requires at least one visual element,
indicating a single, central focal point, but other visual elements may be
included as bystanding characters or scenery elements.
At the rendering level, a frame includes instructions for where in the
panel to place supplied visual elements.

A {\bf panel} is a frame instantiated by specific visual elements.

% Modifier: visual details overlaid on frames and VEs to add semantic
% coherence to the comic, such as floating emotes, facial expressions, motion
% lines, word balloons, and other text.

Finally, a {\bf transition} is (XXX continue)


\subsection{Formal Transition Types}

% XXX format and edit; add sentences...

Moment: keep VEs, change frame and/or modifiers.

Add: introduce a VE that didn't appear in the previous panel (but might
have appeared earlier).

Subtract: remove a VE from the previous panel (and potentially choose a new
frame).

Meanwhile: choose a new frame and only show VEs that didn't appear in the
previous panel, generating new VEs if necessary.

Rendez-vous: choose a new frame and fill it with any combination of
previously-appearing VEs. Generate new VEs only when there aren't enough
previous VEs to fill the frame.


\subsection{Example}

\section{Implementation}

\section{Output}
