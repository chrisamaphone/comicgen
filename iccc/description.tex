\section{System Description}

XXX overview our approach

% XXX how to make this a heading that looks different from a section
% heading?
\subsection{Visual elements, frames, and transitions}

% XXX edit from blog post text

Visual element (VE): unique identifier from an infinite set, mappable to
visually distinct image components, such as anthropomorphic "characters,"
scenery, and geometric shapes.

Frame: a named panel outline dictating a minimum number of visual elements
requires to fill it in, e.g. "give" requires three visual elements (a
giver, a gift, and a giftee). The frame should contain instructions for
visual rendering, e.g. an image with three holes for the spatial positions
of each element.

Panel: a frame with its holes filled by visual elements, and optionally
some additional VEs (e.g. observers carried over from previous panels).

Modifier: visual details overlaid on frames and VEs to add semantic
coherence to the comic, such as floating emotes, facial expressions, motion
lines, word balloons, and other text.

\subsection{Formal Transition Types}

% XXX format and edit; add sentences...

Moment: keep VEs, change frame and/or modifiers.

Add: introduce a VE that didn't appear in the previous panel (but might
have appeared earlier).

Subtract: remove a VE from the previous panel (and potentially choose a new
frame).

Meanwhile: choose a new frame and only show VEs that didn't appear in the
previous panel, generating new VEs if necessary.

Rendez-vous: choose a new frame and fill it with any combination of
previously-appearing VEs. Generate new VEs only when there aren't enough
previous VEs to fill the frame.


\subsection{Example}

\section{Implementation}

\section{Output}
