%================================================================
\section{Future Work}

There are three main avenues that we would like to explore to further develop
this work.

The first avenue is with respect to a refinement of our system's discourse 
model. We would like to refine our model in several key ways, including\ldots
Reformulating the implementation:
\begin{itemize}
\item Connecting panel internals to narrative structure: roles for visual
elements
\item Exchanging visual element palettes along a spectrum of abstraction
\item Reformulating transitions in terms of ``edits'' on previous panels
rather than simply their frame/VE sets -- which VE gets assigned to which
frame position is lost information, for example
\end{itemize}

The second avenue is with respect to the system's expressivity. Currently, our
system cannot reason about the following key aspects\ldots
\begin{itemize}
\item Text and images together
\item More hierarchical structure -- sequences of comic ``pages'';
larger-scale stories
\end{itemize}

The third avenue is with respect to the empirical evaluation of this work.
There are several ways that we could evaluate our system, and each evaluation
would lend different strengths to our approach. 

One candidate evaluation involves analyzing the style and variety of our
comic generator's output; i.e. our system's 
\emph{expressive range}~\cite{smith2010analyzing}. For this, and as
suggested by \citeauthor{smith2010analyzing}, we would need to identify
appropriate metrics for describing the generated output, which ``should be
based on global properties \ldots and ideally should be emergent qualities
from the point of view of the generator.'' A textually-focused candidate 
metric is the number and type of transitions that are generated on average
in a large sample of generated comics. A cognitively-focused candidate
metric is the average number of unique readings that an audience comes up
with for generated comics. Further, these metrics should be evaluated in
the context of the discourse grammar's \emph{cyclomatic complexity}~\cite{mccabe1976complexity}, which in our case is low; such an analysis
will yield insight into the representational power that the grammar has
for generating narrative discourse, relative to the system's overall
computational complexity.

Another candidate evaluation involves analyzing the level of comprehension
that our generated comics afford an audience. While there has been work in
understanding how people read into narratives involving abstract
shapes~\citeeg{heider1944experimental}, this evaluation would be more 
concerned with whether the discourse categories (as discussed by Cohn) that
guide the selection of transitions are recognizable by an audience during
comprehension. \citeA{cohn2015narrative} discusses a methodology through 
which panel discourse categories can be analytically identified; this analysis
would ask whether comic panel categories can be analytically identified by
an audience when they are intentionally selected by our generative system.







