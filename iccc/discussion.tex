\section{Future Work}

To further develop this work, we would like to carry out evaluations of our
system as well as refine it to more accurately reflect leading discourse
theories.

Evaluation:
\begin{itemize}
\item Expressive range analysis~\cite{smith2010analyzing}
\item Narrative comprehension tests
\item Others?
\end{itemize}

Reformulating the implementation:

\begin{itemize}
\item Connecting panel internals to narrative structure: roles for visual
elements
\item Exchanging visual element palettes along a spectrum of abstraction
\item Reformulating transitions in terms of ``edits'' on previous panels
rather than simply their frame/VE sets -- which VE gets assigned to which
frame position is lost information, for example
\end{itemize}

Extending what we can talk about:
\begin{itemize}
\item Text and images together
\item More hierarchical structure -- sequences of comic ``pages'';
larger-scale stories
\end{itemize}


\section{Conclusion}

% presented a small-scale computational system for generating comics down
% to the level of visual elements within panels, as a vector for exploring
% discourse-driven narrative generation
We have presented a small-scale computational system for generating comics,
down to the level of visual elements within panels, informed by leading
narrative discourse theories.
% reiterate takeaway: both top-down and bottom-up theories are needed.
% bottom-up generation enforces relatedness and cohesion (coherence?),
% but top-down generation is needed for global structure.
Our takeaway lesson is that both bottom-up and top-down procedures are
necessary for coherent and cohesive discourse generation: top-down
structure is required for global coherence, while bottom-up information
about the relationships between generated elements are needed for local
cohesion. Our algorithms and implementation offer a promising starting
point for the computational investigation of discourse-driven narrative
theories.



