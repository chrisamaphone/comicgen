%================================================================
\section{Future Work}

There are three main avenues that we would like to explore to further develop
this work.

The first avenue is with respect to a refinement of our system's discourse 
model. We would like to refine our model in several key ways, including\ldots
Reformulating the implementation:
\begin{itemize}
\item Connecting panel internals to narrative structure: roles for visual
elements
\item Exchanging visual element palettes along a spectrum of abstraction
\item Reformulating transitions in terms of ``edits'' on previous panels
rather than simply their frame/VE sets -- which VE gets assigned to which
frame position is lost information, for example
\end{itemize}

The second avenue is with respect to the system's expressivity. Currently, our
system cannot reason about the following key aspects\ldots
\begin{itemize}
\item Text and images together
\item More hierarchical structure -- sequences of comic ``pages'';
larger-scale stories
\end{itemize}

The third avenue is with respect to the empirical evaluation of this work.
There are several ways that we could evaluate our system, and each evaluation
would lend different strength to our approach. One candidate evaluation involves
analyzing our system's \emph{expressive range}~\cite{smith2010analyzing}.

Another candidate evaluation involves analyzing the level of comprehension
that our generated comics afford an audience. While there has been work in
understanding how people read into narratives involving abstract
shapes~\citeeg{heider1944experimental}, this evaluation would be more 
concerned with whether the discourse categories (as discussed by Cohn) that
guide the selection of transitions are recognizable by an audience during
comprehension. \citeA{cohn2015narrative} discusses a methodology through 
which panel discourse categories can be analytically identified; we are 
interested in whether comic panel categories can be analytically identified 
when they are intentionally selected by our generative system.

\note{RCR}{Other evaluation-focused work?}






