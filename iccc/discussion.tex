\section{Future Work}

Evaluation:
\begin{itemize}
\item Expressive range analysis
\item Narrative comprehension tests
\item Others?
\end{itemize}

Other future work, reformulating the implementation:

\begin{itemize}
\item Connecting panel internals to narrative structure: roles for visual
elements
\item Exchanging visual element palettes along a spectrum of abstraction
\item Reformulating transitions in terms of ``edits'' on previous panels
rather than simply their frame/VE sets -- which VE gets assigned to which
frame position is lost information, for example
\item More hierarchical structure -- sequences of comic ``pages'';
larger-scale stories
\end{itemize}


\section{Conclusion}

% presented a small-scale computational system for generating comics down
% to the level of visual elements within panels, as a vector for exploring
% discourse-driven narrative generation
%
% reiterate takeaway: both top-down and bottom-up theories are needed.
% bottom-up generation enforces relatedness and cohesion (coherence?),
% but top-down generation is needed for global structure.
%
%a top-down grammar helps, but to maintain relatedness
% across panels, a formalization of bottom-up transitions was helpful, too.
